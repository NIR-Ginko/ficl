%!TEX encoding = UTF-8 Unicode
\chapter{Release History}
\section{Release Plan}
\begin{itemize}[noitemsep]
	\item 4.3.0
		\subitem Add manpage in ru\_RU.UTF-8.
		\subitem Add FreeBSD patches to the distribution.
		\subitem Improve LaTeX documentation sources.
\end{itemize}


\section{Version 4.2.0}
I don't know what was changed for 4.1.0 so here is the list of the
work I've done:
\begin{itemize}[noitemsep]
	\item Manpage added in en\_US.UTF-8. Please translate it into
		other languages as well.
	\item Documentation converted into LaTeX so you can convert it
		into any format you like - HTML or PDF for example.
	\item Ficl now builds with CMake (CMakeLists.txt added) but
		I also support classic Makefile.
	\item Ficl now understands POSIX shebang.
	\item Ficl can now run in quiet mode (for scripting) when
		ran like 'ficl -q somescript.fth'.
\end{itemize}
All the changes are public domain (or any license you like).


\section*{Version 4.0.31}
\begin{itemize}[noitemsep]
	\item First official release of new engine as Ficl 4! Hooray!

	\item ficlDictionarySee() now takes a ficlCallback, so it knows
	where to print to. This is because ficlWin only sets a per-VM
	callback, which should work.

	\item ficlSystemCreate() now passes in the system correctly into
	the dictionaries it creates, which lets dictionaries know what
	system they're a part of.

	\item ficlCompatibility: Forgot to add the errorTextOut to the
	ficl\_system structure (though I'd remembered to add it to the
	ficl\_vm structure). This caused the ficl\_system members after
	textOut to not line up with their equivalent ficlSystem members,
	which did bad things. (The bad thing in particular was calling
	ficlDictionaryResetSearchOrder() resulted in diddling the
	vm-\textgreater link member, which strangely enough resulted in
	double-freeing the stacks.)

	\item Added ficlStackWalk(), which walks a stack from top to
	bottom and calls your specified callback with each successive
	element. Cleaned up stack-printing functions as a result.

	\item Changed MULTICALL so you can explicitly specify the vtable.

	\item Changed XClasses so it explicitly specifies the vtable for
	non-virtual classes. This means you can now call a virtual
	method when you've SUPERed an object and you'll get the method
	you wanted.

	\item XClasses improvement: when removing a thunked method,
	remove the thunk variable too. Added xClass.removeMember() to
	support this.

	\item XClasses now generates runtime stack-check code (\_DEBUG
	only) for functions thunked from C to FICL.

	\item FICL\_WANT\_PLATFORM is now 0 by default. It is now set
	to 1 in the appropriate \textit{ficlplatform/*.h}.

	\item \textit{softcore/win32.fr} ENVIRONMENT? COMPARE needed to
	be case-insensitive.

	\item Whoops! Setting FICL\_PLATFORM\_2INTEGER to 0 didn't
	compile. It now does, and works fine, as proved by the ansi
	platform.

	\item Another whoops: \textit{contrib/xclasses/xclasses.py}
	assumed that " (a prefix version of S") defined. Switched to S",
	which is safer.
\end{itemize}


\section*{Version 4.0.30}
\begin{itemize}[noitemsep]
	\item Cleaned up some FICL\_ definitions. Now all FICL\_HAVE\_*
	constants (and some other odds and ends) have been
	moved to FICL\_PLATFORM\_.

	\item Whoops! Setting FICL\_PLATFORM\_2INTEGER to 0 didn't compile.
	It now does, and works fine, as proved by the "ansi" platform.

	\item Another whoops: \textit{contrib/xclasses/xclasses.py}
	assumed that " (a prefix version of S") defined. Switched to S",
	which is safer.

	\item Added ficlDictionarySetConstantString(). 'Cause I needed
	it for:

	\item Removed the "WIN32" ENVIRONMENT? setting, and added
	"FICL\_PLATFORM\_OS" and "FICL\_PLATFORM\_ARCHITECTURE" in its
	place. These are both strings. Updated
	\textit{softcore/win32.fr} to match.

	\item Compatibility: improved ficlTextOut() behavior. It makes
	life slightly less convenient for some users, but should be an
	improvement overall. The change: ficlTextOut() is now a
	compatibility-layer function that calls straight through to
	vmTextOut(). Lots of old code calls ficlTextOut() (naughty!).
	It's now explicit that you must set the textOut function by hand
	if you use a custom one... which is a good habit to get in to
	anyway.

	\item Improved the documentation regarding upgrading,
	\textit{ficllocals.h}, and compile-time constants.

	\item Fixed \textit{doc/source/generate.py} so it gracefully
	fails to copy over read-only files.

	\item Got rid of every \# ifdef in the sources. We now
	consistently use \# if defined() everywhere. Similarly, got rid
	of all platform-switched \# if code (except for the
	compatibility layer, sigh).
\end{itemize}


\section*{Version 4.0.29}
\begin{itemize}[noitemsep]
	\item Documentation totally reworked and updated.

	\item oldnames renamed to compatibility. And improved, so that
	now FICL 4 is basically a drop-in replacement for FICL 3.
\end{itemize}


\section*{Version 4.0.28}
\begin{itemize}[noitemsep]
	\item Did backwards-compatibility testing. FICL now drops in,
	more or less, with all the old FICL-3.03-using projects I had
	handy.

	\item Got FICL compiling and running fine on Linux.

	\item Weaned LZ77 code from needing htonl()/ntohl().

	\item Moved all the primitives defined in \textit{"testmain.c"}
	to their own file, \textit{"extras.c"}, and gave it its own
	global entry point.

	\item Renamed \textit{"testmain.c"} to just plain \textit{"main.c"}.

	\item Renamed \textit{"softwords"} directory to
	\textit{"softcore"}. More symmetrical.

	\item Renamed \textit{"softcore/softcore.bat"} to
	\textit{"make.bat"}. Added support for "CLEAN".
\end{itemize}


\section*{Version 4.0.27}
\begin{itemize}[noitemsep]
	\item Added runtime jump-to-jump peephole optimization in the
	new switch-threaded VM.

	\item Fixed INCLUDE-FILE so it rethrows an exception in the
	subordinate evaluation.

	\item Added a separate errorOut function to ficlCallback(), so
	under Windows you can have a jolly popup window to rub your nose
	in your failings.
\end{itemize}


\section*{Version 4.0.26}
\begin{itemize}[noitemsep]
	\item Namespace policing complete. There are now no external
	symbols which do not start with the word ficl.

	\item Removed ficlVmExec(), renamed ficlVmExecC() to
	ficlVmExecuteString(), changed it to take a ficlString(). This
	is deliberate subterfuge on my part; I suspect most people who
	currently call ficlVmExec() / ficlVmExecC() should be calling
	ficlVmEvaluate().
\end{itemize}


\section*{Version 4.0.25}
\begin{itemize}[noitemsep]
	\item First pass at support for "oldnames", and namespace
	policing.
\end{itemize}


\section*{Version 4.0.23}
First alpha release of FICL 4.0 rewrite. Coded, for better or for worse,
by Larry Hastings. FICL is smaller, faster, more powerful, and easier to
use than ever before. (Or your money back!)
\begin{itemize}[noitemsep]
	\item Rewrote FICL's virtual machine; FICL now runs nearly 3x
	faster out-of-the-box. The new virtual machine is of the "big
	switch statement" variety.

	\item Renamed most (probably all) external FICL functions and
	data structures. They now make sense and are (gasp!) consistent.

	\item Retooled double-cell number support to take advantage of
	platforms which natively support double-cell-sized integers.
	(Like most modern 32-bit platforms.)

	\item Locals and VALUEs are now totally orthogonal; they can be
	single- or double-cell, and use the float or data stack. TO
	automatically supports all variants.

	\item The "softcore" words can now be stored compressed, with a
	(current) savings of 11k. Decompression is nigh-instantaneous.
	You can choose whether or not you want softcore stored
	compressed at compile-time.

	\item Reworked Win32 build process. FICL now builds
	out-of-the-box on Win32 as a static library, as a DLL, and as a
	command-line program, in each of the six possible runtime
	variants (Debug, Release x Singlethreaded, Multithreaded,
	Multithreaded DLL).
\end{itemize}


\section*{Version 3.03}
\begin{itemize}[noitemsep]
	\item Bugfix for floating-point numbers. Floats in compiled
	code were simply broken.

	\item New words: random and seed-random

	\item Bugfix: included never closed its file.

	\item Bugfix: include was not IMMEDIATE.

	\item Un-hid the OO words parse-method, lookup-method, and
	find-method-xt, as there are perfectly legitimate reasons why
	you might want to use them.

	\item Changed the prefix version of .( to be IMMEDIATE too.

	\item Fixed comment in Python softcore builder.

	\item Put the \textit{doc} directory back in to the
	distribution. (It was missing from 3.02... where'd it go?)
\end{itemize}


\section*{Version 3.02}
\begin{itemize}[noitemsep]
	\item Added support for nEnvCells (number of environment cells)
	to FICL\_SYSTEM\_INFO.

	\item Consolidated context and pExtend pointers of
	FICL\_SYSTEM—VM's pExtend pointer is initialized from the copy
	in FICL\_SYSTEM upon VM creation.

	\item Added ficl-robust environment variable.

	\item Added FW\_ISOBJECT word type.

	\item Bugfix: environment? was ignoring the length of the
	supplied string.

	\item Portability cleanup in \textit{fileaccess.c}.

	\item Bugfix in ficlParsePrefix: if the prefix dictionary isn't
	in the wordlist, the word being examined cannot be a prefix, so
	return failure.

	\item SEE improvements: SEE (and consequently DEBUG) have
	improved source listings with instruction offsets.

	\item It's turned off with the preprocessor, but we have the
	beginnings of a switch-threaded implementation of the inner loop.

	\item Added objectify and ?object for use by OO infrastructure.

	\item my=[ detects object members (using ?object) and assumes
	all other members leave class unchanged.

	\item Removed MEMORY-EXT environment variable (there is no such
	wordset).

	\item Ficlwin changes:
		\subitem Ficlwin character handling is more robust

		\subitem Ficlwin uses multi-system constructs (see
		\textit{ficlthread.c})

	\item Documentation changes:
		\subitem Corrected various bugs in docs.

		\subitem Added FICL-ized version of JV Noble's Forth
		Primer

		\subitem FICL OO tutorial expanded and revised. Thanks
		to David McNab for his demo and suggestions.
\end{itemize}


\section*{Version 3.01}
\begin{itemize}[noitemsep]
	\item Major contributionss by Larry Hastings (larry@hastings.org):
		\subitem FILE wordset (\textit{fileaccess.c})

		\subitem ficlEvaluate wrapper for ficlExec

		\subitem ficlInitSystemEx makes it possible to bind
		selectable properties to VMs at create time

		\subitem Python version of softcore builder
		\textit{ficl/softwords/softcore.py}
	\item Environment contains ficl-version (double)

	\item ?number handles trailing decimal point per DOUBLE wordset
	spec

	\item Fixed broken .env (thanks to Leonid Rosin for spotting
	this goof)

	\item Fixed broken floating point words that depended on
	evaluation order of stack pops.

	\item env-constant

	\item env-2constant

	\item dictHashSummary is now commented out unless
	FICL\_WANT\_FLOAT (thanks to Leonid Rosin again)

	\item Thanks to David McNab for pointing out that .( should be
	IMMEDIATE. Now it is.
\end{itemize}


\section*{Version 3.00a}
\begin{itemize}[noitemsep]
	\item Fixed broken \textit{oo.fr} by commenting out vcall stuff
	using FICL\_WANT\_VCALL. Vcall is still broken.
\end{itemize}


\section*{Version 3.00}
\begin{itemize}[noitemsep]
	\item Added pSys parameter to most ficlXXXX functions for
	multiple system support. Affected functions:
		\subitem dictLookupLoc renamed to ficlLookupLoc after
		addition of pSys param

		\subitem ficlInitSystem returns a FICL\_SYSTEM*

		\subitem ficlTermSystem

		\subitem ficlNewVM

		\subitem ficlLookup

		\subitem ficlGetDict

		\subitem ficlGetEnv

		\subitem ficlSetEnv

		\subitem ficlSetEnvD

		\subitem ficlGetLoc

		\subitem ficlBuild
	\item Fixed off-by-one bug in ficlParsePrefix

	\item FICL parse-steps now work correctly - mods to interpret()

	\item Made \textit{tools.c}:isAFiclWord more selective

	\item Tweaked makefiles and code to make gcc happy under linux

	\item Vetted all instances of LVALUEtoCELL to make sure they're
	working on CELL sized operands (for 64 bit compatibility)
\end{itemize}


\section*{Version 2.06}
\begin{itemize}[noitemsep]
	\item Debugger changes:
		\subitem New debugger command "x" to execute the rest
		of the command line as FICL

		\subitem New debugger command "l" lists the source of
		the innermost word being debugged

		\subitem If you attempt to debug a primitive, it gets
		executed rather than doing nothing

		\subitem R.S displays the stack contents symbolically

		\subitem Debugger now operates correctly under ficlwin,
		although ficlwin's key handling leaves a lot to be
		desired.

		\subitem EE listing enhanced for use with the debugger

	\item Added Guy Carver's changes to \textit{oo.fr} for VTABLE
	support

	\item \textit{float.c} words f\textgreater  and \textgreater f
	to move floats to and from the param stack, analogous
	to \textgreater r and r\textgreater

	\item LOOKUP - Surrogate precompiled parse step for
	ficlParseWord (this step is hard coded in INTERPRET)

	\item License text at top of source files changed from LGPL to
	BSD by request

	\item Win32 console version now handles exceptions more
	gracefully rather than crashing - uses win32 structured
	exception handling.

	\item Fixed BASE bug from 2.05 (was returning the value rather
	than the address)

	\item Fixed ALLOT bug - feeds address units to dictCheck, which
	expects Cells. Changed dictCheck to expect AU.

	\item Float stack display word renamed to f.s from .f to be
	consistent with r.s and .s
\end{itemize}


\section*{Version 2.05}
\subsection*{General}
\begin{itemize}[noitemsep]
	\item HTML documentation extensively revised

	\item Incorporated Alpha (64 bit) patches from the FreeBSD team.

	\item Split SEARCH and SEARCH EXT words from \textit{words.c}
	to search.c

	\item 2LOCALS defined in Johns Hopkins local syntax now lose the
	first '2:' in their names.

	\item Simple step debugger (see \textit{tools.c})

	\item The text interpreter is now extensible - this is
	accomplished through the use of ficlAddParseStep().
	FICL\_MAX\_PARSE\_STEPS limits the number of parse steps
	(default: 8). You can write a precompiled parse step (see
	ficlParseNumber) and append it to the chain, or you can write
	one in FICL and use ADD-PARSE-STEP to append it. Default parse
	steps are initialized in ficlInitSystem. You can list the parse
	steps with parse-order ( -- ).

	\item There is now a FICL\_SYSTEM structure. This is a
	transitional release - version 3.0 will alter several API
	prototypes to take this as a parameter, allowing multiple
	systems per process (and therefore multiple dictionaries). For
	those who use FICL under a virtual memory OS like Linux or
	Windows NT, you can just create multiple FICL processes (not
	threads) instead and save youself the wait.

	\item Fixes for improved command line operation in
	\textit{testmain.c} (Larry Hastings)

	\item Numerous extensions to OO facility, including a new allot
	methods, ability to catch method invocations (thanks to Daniel
	Sobral again)

	\item Incorporated Alpha (64 bit) patches contributed by Daniel
	Sobral and the freeBSD team Ficl is now 64 bit friendly! UNS32
	is now FICL\_UNS.

	\item Split SEARCH and SEARCH EXT words from \textit{words.c}
	to \textit{search.c}

	\item ABORT" now complies with the ANS (-2 THROWs)

	\item Floating point support contributed by Guy Carver
	(Enable FICL\_WANT\_FLOAT in \textit{sysdep.h}).

	\item Win32 vtable model for objects (Guy Carver)

	\item Win32 dll load/call suport (Larry Hastings)

	\item Prefix support (Larry Hastings) (\textit{prefix.c}
	\textit{prefix.fr}
	FICL\textunderscore EXTENDED\textunderscore PREFIX) makes
	it easy to extend the parser to recognize prefixes like 0x
	and act on them. Use show-prefixes to see what's defined.

	\item Cleaned up initialization sequence so that it's all in
	ficlInitSystem, and so that a VM can be created successfully
	before the dictionary is created
\end{itemize}


\subsection*{Bug fixes}
\begin{itemize}[noitemsep]
	\item ABORT" now works correctly
	\item REFILL works better
	\item ALLOT's use of dictCheck corrected
\end{itemize}


\subsection*{New words}
\begin{itemize}[noitemsep]
	\item 2r@ 2r\textgreater 2\textgreater r (CORE EXT)

	\item 2VARIABLE (DOUBLE)

	\item ORDER now lists wordlists by name

	\item .S now displays all stack entries on one line, like a
	stack comment

	\item wid-get-name   given a wid, returns the address and count
	of its name. If no name, count is 0

	\item wid-set-name  set optional wid name pointer to the \\0
	terminated string address specified.

	\item ficl-named-wordlist creates a ficl-wordlist and names it.
	This is now used in vocabulary and ficl-vocabulary

	\item last-word  returns the xt of the word being defined or
	most recently defined.

	\item q@ and q! operate on quadbyte quantities for 64 bit
	friendliness
\end{itemize}


\subsection*{New OO stuff}
\begin{itemize}[noitemsep]
	\item ALLOT (class method)

	\item ALLOT-ARRAY (class method)

	\item METHOD define method names globally

	\item MY=\textgreater  early bind a method call to "this" class

	\item MY=[ ] early bind a string of method calls to "this" class
	and obj members

	\item C-\textgreater  late bind method invocation with CATCH

	\item Metaclass method resume-class and instance word
	suspend-class create mutually referring classes. Example in
	\textit{string.fr}

	\item Early binding words are now in the instance-vars wordlist,
	not visible unless defining a class.

	\item Support for refs to classes with VTABLE methods
	(contributed by Guy Carver). Guy writes:

	My next favorite change is a set of VCALL words that allow me to
	call C++ class virtual methods from my FORTH classes. This is
	accomplished by interfacing with the VTABLE of the class. The
	class instance currently must be created on the C++ side. C++
	places methods in the VTABLE in order of declaration in the
	header file. To use this in FICL one only needs to ensure that
	the VCALL: declerations occur in the same order. I use this
	quite a bit to interface with the C++ classes. When I need
	access to a method I make sure it is virtual (Even if it
	ultimately will not be). I use Visual C++ 6.0 and have not
	tested this under any other compiler but I believe VTABLE
	implementation is standard.

	Here is an example of how to use VCALL:

	\textbf{C++ class declaration}
\begin{lstlisting}[frame=single]
class myclass {
public:
	myclass();
	virtual ~myclass();

	virtual void Test( int iParam1 );
	virtual int Test( int iParam1, char cParam2 );
	virtual float Test();
};
\end{lstlisting}
	\textbf{ficl class declaration}
\begin{lstlisting}[frame=single]
object subclass myfclass hasvtable   \ hasvtable adds 4 to the offset to
\  accommodate for the VTABLE pointer.
0 VCALL: Destructor()      \ VCALL: ( ParamCount -- )
1 VCALL: Test(int)         \ Test takes 1 int parameter.
2 VCALLR: iTest(int,char)  \ iTest takes 2 parameters and returns an int.  
0 VCALLF: fTest()          \ fTest takes no parameters and returns a float.
end-class

MyCAddress                 \ Primitive to return a pointer to a "myclass" instance.
myfclass -> ref dude       \ This makes the MyCAddress pointer a myfclass
\  instance with the name "dude".
1234 dude -> Test(int)     \ Calls the virtual method Test.
1234 1 dude -> iTest(int,char) .  \ Calls iTest and emits the returned int.
dude -> fTest() f.         \ Calls fTest and emits the returned float.
\end{lstlisting}
\end{itemize}


\section*{Version 2.04}
\subsection*{ficlwin}
\begin{itemize}[noitemsep]
	\item Catches exceptions thrown by VM in ficlThread (0 @ for
	example) rather than passing them off to the OS.
\end{itemize}


\subsection*{FICL bugs vanquished}
\begin{itemize}[noitemsep]
	\item Fixed leading delimiter bugs in s" ." .( and (
	(reported by Reuben Thomas)

	\item Makefile tabs restored (thanks to Michael Somos)

	\item ABORT" now throws -2 per the DPANS (thanks to Daniel
	Sobral for sharp eyes again)

	\item ficlExec does not print the prompt string unless
	(source-id == 0)

	\item Various fixes contributed by the FreeBSD team.
\end{itemize}


\subsection*{FICL enhancements}
\begin{itemize}[noitemsep]
	\item \textit{words.c}: modified ficlCatch to use vmExecute and
	vmInnerLoop (request of Daniel Sobral) Added vmPop and vmPush
	functions (by request of Lars Krueger) in \textit{vm.c}. These
	are shortcuts to the param stack. (Use LVALUEtoCELL to get
	things into CELL form)

	\item Added function vmGetStringEx with a flag to specify
	whether or not to skip lead delimiters

	\item Added non-std word: number?

	\item Added CORE EXT word AGAIN (by request of Reuben Thomas)

	\item Added double cell local (2local) support

	\item Augmented Johns Hopkins local syntax so that locals whose
	names begin with char 2 are treated as 2locals (OK - it's goofy,
	but handy for OOP)

	\item C-string class revised and enhanced - now dynamically
	sized

	\item C-hashstring class derived from c-string computes hashcode
	too.
\end{itemize}


\section*{Version 2.03}
This is the first version of FICL that includes contributed code. Thanks
especially to Daniel Sobral, Michael Gauland for contributions and bug
finding.

\subsection*{New words}
\begin{itemize}[noitemsep]
	\item clock              (FICL)
	\item clocks/sec         (FICL)
	\item dnegate            (DOUBLE)
	\item ms                 (FACILITY EXT - replaces MSEC ficlWin only)
	\item throw              (EXCEPTION)
	\item catch              (EXCEPTION)
	\item allocate           (MEMORY)
	\item free               (MEMORY)
	\item resize             (MEMORY)
	\item within             (CORE EXT)
	\item alloc              (class method)
	\item alloc-array        (class method)
	\item free               (class method)
\end{itemize}


\subsection*{Bugs Fixed}
\begin{itemize}[noitemsep]
	\item Bug fix in isNumber(): used to treat chars between 'Z'
	and 'a' as valid in base 10... (harmless, but weird)

	\item ficlExec pushes the ip and interprets at the right times
	so that nested calls to ficlExec behave the way you'd expect
	them to.

	\item evaluate respects count parameter, and also passes
	exceptional return conditions back out to the calling instance
	of ficlExec.

	\item VM\_QUIT now clears the locals dictionary in ficlExec.
\end{itemize}


\subsection*{Ficlwin Enhancements}
\begin{itemize}[noitemsep]
	\item File Menu: recent file list and Open now load files.

	\item Text ouput function is now faster through use of string
	caching. Cache flushes at the end of each line and each time
	ficlExec returns.

	\item Edit/paste now behaves more reasonably for text.
	File/open loads the specified file.

	\item Registry entries specify dictionary and stack sizes,
	default window placement, and whether or not to create a
	splitter for multiple VMs. See
	\textit{HKEY\_CURRENT\_USER/Software/CodeLab/ficlwin/Settings}
\end{itemize}


\subsection*{FICL Enhancements}
\begin{itemize}[noitemsep]
	\item This version includes changes to make it 64 bit friendly.
	This unfortunately meant that I had to tweak some core data
	types and structures. I've tried to make this transparent to 32
	bit code, but a couple of things got renamed. INT64 is now
	DPINT. UNS64 is now DPUNS. FICL\_INT and FICL\_UNS are synonyms
	for INT32 and UNS32 in 32 bit versions, but a are obsolescent.
	Please use the new data types instead. Typed stack operations
	on INT32 and UNS32 have been renamed because they operate on
	CELL scalar types, which are 64 bits wide on 64 bit systems.
	Added BITS\_PER\_CELL, which has legal values of 32 or 64.
	Default is 32.

	\item \textit{ficl.c}: Added ficlExecXT() - executes an xt
	completely before returning, passing back any exception codes
	generated in the process. Normal exit code is VM\_INNEREXIT.

	\item \textit{ficl.c}: Added ficlExecC() to operate on counted
	strings as opposed to zero terminated ones.

	\item ficlExec pushes ip and executes interpret at the right
	times so that nested calls to ficlExec behave the way you'd
	expect them to.

	\item ficlSetStackSize() allows specification of stack size at
	run-time (affects subsequent invocations of ficlNewVM()).

	\item \textit{vm.c}: vmThrow() checks for (pVM-\textgreater
	pState != NULL) before longjmping it. vmCreate nulls this
	pointer initially.

	\item EXCEPTION wordset contributed by Daniel Sobral of FreeBSD

	\item MEMORY-ALLOC wordset contributed by Daniel Sobral, too.
	Added class methods alloc and alloc-array in
	\textit{softwords/oo.fr} to allocate objects from the heap.

	\item Control structure match check upgraded (thanks to Daniel
	Sobral for this suggestion). Control structure mismatches are
	now errors, not warnings, since the check accepts all
	syntactically legal constructs.

	\item Added vmInnerLoop() to \textit{vm.h}. This function/macro
	factors the inner  interpreter out of ficlExec so it can be used
	in other places. Function/macro behavior is conditioned on
	INLINE\_INNER\_LOOP in \textit{sysdep.h}. Default: 1 unless
	\_DEBUG is set. In part, this is because VC++ 5 goes apoplectic
	when trying to compile it as a function. See comments in
	\textit{vm.c}

	\item EVALUATE respects the count parameter, and also passes
	exceptional return conditions back out to the calling instance
	of ficlExec.

	\item VM\_QUIT clears locals dictionary in ficlExec()

	\item Added Michael Gauland's ficlLongMul and ficlLongDiv and
	support routines to \textit{math64.c} and \textit{math64.h}.
	These routines are coded in C, and are compiled only if
	PORTABLE\_LONGMULDIV == 1 (default is 0).

	\item Added definition of ficlRealloc to \textit{sysdep.c}
	(needed for memory allocation wordset). If your target OS
	supports realloc(), you'll probably want to redefine ficlRealloc
	in those terms. The default version does ficlFree followed by
	ficlMalloc.

	\item \textit{testmain.c}: Changed gets() in testmain to fgets()
	to appease the security gods.

	\item \textit{testmain.c}: msec renamed to ms in line with the
	ANS

	\item \textit{softcore.pl} now removes comments \& spaces at the
	start and end of lines. As a result: sizeof (softWords) == 7663
	bytes (used to be 20000)  and consumes 11384 bytes of dictionary
	when compiled

	\item Deleted license paste-o in \textit{readme.txt} (oops).
\end{itemize}


\section*{Version 2.02}
New words:
\begin{itemize}[noitemsep]
	\item marker        (CORE EXT)
	\item forget       (TOOLS EXT)
	\item forget-wid        (FICL)
	\item ficl-wordlist     (FICL)
	\item ficl-vocabulary   (FICL)
	\item hide              (FICL)
	\item hidden            (FICL)
	\item Johns Hopkins local variable syntax (as best I can determine)
\end{itemize}


Bugs Fixed
\begin{itemize}[noitemsep]
	\item forget now adjusts the dictionary pointer to remove the
	name of the word being forgotten (name chars come before the
	word header in FICL's dictionary)

	\item :noname used to push the colon control marker and its
	execution token in the wrong order

	\item source-id now behaves correctly when loading a file.

	\item refill returns zero at EOF (Win32 load). Win32 load
	command continues to be misnamed. Really ought to be called
	included, but does not exactly conform to that spec either
	(because included expects a string signature on the stack,
	while FICL's load expects a filename upon invocation). The
	"real" LOAD is a BLOCK word.
\end{itemize}

Enhancements
\begin{itemize}[noitemsep]
	\item dictUnsmudge no longer links anonymous definitions into
	the dictionary

	\item oop is no longer the default compile wordlist at startup,
	nor is it in the search order. Execute also oop definitions to
	use FICL OOP.

	\item Revised \textit{oo.fr} extensively to make more use of
	early binding

	\item Added meta - a constant that pushes the address of
	metaclass. See \textit{oo.fr} for examples of use.

	\item Added classes: c-ptr  c-bytePtr  c-2bytePtr  c-cellPtr
	These classes model pointers to non-object data, but each knows
	the size of its referent.
\end{itemize}


\section*{Version 2.01}
\begin{itemize}[noitemsep]
	\item Bug fix: (local) used to leave a value on the stack
	between the first and last locals declared. This value is now
	stored in a static.

	\item Added new local syntax with parameter re-ordering. See
	description below. (No longer compiled in version 2.02, in favor
	of the Johns Hopkins syntax)
\end{itemize}


\section*{Version 2.0}
\begin{itemize}[noitemsep]
	\item New ANS Forth words: TOOLS and part of TOOLS EXT, SEARCH
	and SEARCH EXT, LOCALS and LOCALS EXT word sets, additional
	words from CORE EXT, DOUBLE, and STRING. (See the function
	ficlCompileCore in \textit{words.c} for an alphabetical list by
	word set).

	\item Simple USER variable support - a user variable is a
	virtual machine instance variable. User variables behave as
	VARIABLEs in all other respects.

	\item Object oriented syntax extensions (see below)

	\item Optional stack underflow and overflow checking in many
	CORE words (enabled when FICL\_ROBUST \textgreater = 2)

	\item Various bug fixes
\end{itemize}

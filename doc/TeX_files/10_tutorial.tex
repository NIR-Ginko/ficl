%!TEX encoding = UTF-8 Unicode
\chapter{Tutorial}
\section{FICL - an Embeddable Command Language Interpreter}
FORTH for the rest of us.

Glue languages like Perl, Tcl and Python are popular because they
help you to get the results quickly. They make quick work of
problems that are often tedious to code in C or C++, and they can
work with code written in other languages. People often miss
this last benefit because Perl or Python usually run standalone.
Tcl, on the other hand, was designed from scratch as an extension
language and is relatively simple to insert into another program.
All of these languages were designed to work with mainstream
operating systems, so they require lots of memory, a file system,
and other resources that are commonplace on a modern PC or
workstation. In this article, I'll describe an interpreter that has
a system interafe similar in spirit to Tcl, but is specifically
designed for embedded systems with minimal resources. The syntax is
ANS FORTH, so I've called in FICL, or Forth-Inspired Command Language.

FICL is FORTH but you don't have to be a rabid FORTH zealot to use it.
In fact, you can do useful work without knowing any FORTH at all.
On the other hand, you can learn enough in half an hour to do useful
work. Read on and I'll show you. (From here on I'll use words FORTH
and FICL interchangeably.)

Why not write your own command interpreter? While tempting, this
approach has its disadvantages. For one thing, you usually wind up
with something that has line-oriented syntax where the command is at
the beginning and any arguments come afterwards, and you can do
one command per line. This is enough in many cases, and FICL can
do this for you, too. But what if your customer asks for a behavior
that's not hard-coded in the interpreter? Ficl is a small but
complete programming language, not just a command interpreter, so
you can hack together a prototype for that new behavior. If
communication overhead os too high, consider downloading common
groups of commands as stored programs, or "words" in FORTH, and you
can invoke them with a single alias. Because FICL is interactive,
you can use it to help troubleshoot problems. One more advantage of
using FICL is that you can find several good tutorials for free
on the Web, so you save the time on documentation and training:
you only need to document your extensions to the language.

As I implied above, people who do not program for a living can
do useful work with FICL. Your friendly neghborhood electrical
engineer can bench test new hardware quite easily using FICL,
saving your time to add features or fix your own bugs (not that
there will be any). You can use a minimal FICL system to do
rapid prototyping on new hardware, getting new features to a
demonstrable state normally is much less time than would be
required with a compiled language.


\section{Using the Language}
\subsection{Enough FICL Syntax to Make You Dangerous}
\textbf{First Rule of FICL}: Use spaces to separate everything. FORTH is
very simple-minded about parsing its input: it looks for space-delimited
tokens and tries to interpret them one by one.

\textbf{Second Rule of FICL}: If it's not a word, try to make it a
number. If that dpesn't work - it's an error. A word is a named piece of
code (like a function or subroutine) that may also own some data. Words
are organized into a list called a dictionary. For each token in the
input stream, the interpreter tries to find a word with the same name
in the dictionary. If successful, the interpreter will execute the word.
Otherwise, the interpreter will convert the token to a number. If this
fails, you get an error message. By the way, the Second Rule means you
that you can do Evil Things like redefining your favorite number.

\textbf{Third Rule of FICL}: Words find their arguments on a stack. The
interpreter pushes numbers onto the stack automatically. The language
does not have the equivalent of a function prototype, so FORTH
programmers use comments to show the state of the stack before and
after execution of a word. For example:
\begin{lstlisting}[frame=single]
+ ( a b -- c)
\end{lstlisting}
indicates that the word "+" consumes two values from the stack (a and b)
and leaves the third (c, the sum). By the way, an open paren followed by
a space tells the interpreter to treat everything up to the next close
paren as a comment. You can comment to the end of the line with a
backslash character followed by a whitespace.

Here's how you add two numbers in FORTH (if you've ever used an RPN
calculator, you will be in familiar territory):
\begin{lstlisting}[frame=single]
2 3 + .
\end{lstlisting}
The interpreter pushes the number 2 and number 3 then executes the word
"+" and finally executes the word ".". This prints the top value on the
stack, which I hope is five.

There are really two stack in FORTH: one stores return addresses and the
other stores parameters. We'll refer to the parameter stack simply as
"the stack" from now on.

FICL is not case sensitive. Word names can be up to 255 characters long,
but only the first 31 characters will be stored. All words are stored in
a linked list called the dictionary. There exist words to allot space in
the dictionary so that (other) words can have arbitary size data areas.

FICL has these main data structures:
\begin{itemize}[noitemsep]
	\item A virtual machine stores one execution context, and would
	typically map to a thread.

	\item Each virtual machine has two stacks: one for parameters
	and the other for return addresses.

	\item A word binds name to code, and optionally, data.

	\item The dictionary is a list of all words of the system.
\end{itemize}
See section 5 for more FICL syntax.


\subsection{Porting FICL}
As shown in Example 1, it only takes a few lines of code to hook FICL
into your system: initialize the FICL system data structures with a call
ficlInitSystem, and create one or more virtual machines using ficlNewVM.
After that , you simply feed blocks of text to the virtual machine from
an I/O device, a file, or stored strings using ficlExec. You can tear
down memory allocated to FICL using ficlTermSystem.

Example 1:
\begin{lstlisting}[frame=single]
FICL_VM *pVM;

ficlInitSystem(10000); /* Create a 10000 cell dictionary. */
pVM = ficlNewVM();

for (;;) {
	int ret;
	gets(in);
	ret = ficlExec(pVM, in);
	if (VM_USEREXIT == ret) {
		ficlTermSystem();
		break;
	}
}
\end{lstlisting}
FICL requires ANSI C compiler and its runtime library to build and
execute. Porting to a new CPU involves editing two files:
\textit{sysdep.c} and \textit{sysdep.h}. The header file contains macros
that control build properties of FICL, and macros that insulate the
implementation from differences among compilers. FICL interfaces to the
operating system via four functions:
\begin{itemize}[noitemsep]
	\item ficlMalloc and ficlFree map closely to the standard C
	malloc and free functions, but they act as a choke-point for
	FICL memory management in case your OS has specialized functions
	for this purpose or you've rolled your own.

	\item ficlLockDictionary provides exclusive access to the
	dictionary in multithreaded implementations. If you only intend
	to make one FICL virtual machine, this function can be empty.

	\item ficlTextOut is like standard C's puts, except that it
	takes a virtual machine pointer and one additional parameter to
	indicate whether to terminate the string with a newline sequence.
\end{itemize}
Because FICL is a 32-bit FORTH, the language requires some 64-bit math.
There are two unsigned primitives in \textit{sysdep.c} that handle this.
One function multiplies two 32-bit values to yield a 64-bit result, and
the other divides a 64-bit value by a 32-bit value to return a 32-bit
quotinent and remainder. These are usually simple to implement as inline
assembly for a 32-bit CPU (see Intel 386 example in the source). I was
too lazy to come up with a generic version in C. If you're lazy too, you
can cludge these functions to use only the low 32 bits of the 64 bit
parameter and be safe - so long as you avoid multiplying and dividing
really big numbers.

Memory requirements of the code vary by proessor. The dictionary is the
largest RAM-resident structure. The word set that comes with the source
requires requires fewer than 1000 cells or 4K bytes. Stacks default to
128 cells (512 bytes) each, so you can fit a useful implementation into
8K bytes RAM plus code space (which can be in ROM). Use
\textit{testmain.c} as a guide to installing FICL system and one or more
virtual machines into your code. You do not need to include
\textit{testmain.c} in your build.


\subsection{Roll Your Own Extensions in C}
You can extend the language with words that are specific to your
application, written in C, FORTH, or in mixture of C and FORTH. Use
ficlBuild function to bind a C function to a name in the dictionary.
Functions that implement FICL words take one parameter: a pointer to a
FICL\_ VM. This pointer refers to the running virtual machine in whose
context the word executes. The files \textit{words.c} and \textit{main.c}
have (literally) hundreds of examples of words coded in C. Example 2
shows a function that interfaces FICL to the Win32 "chdir" service:
\begin{lstlisting}[frame=single]
/*
** FICL interface to _chdir (Win32)
** Gets a newline (or NULL) delimited string from the input
** and feeds it to the Win32 chdir function...
** Usage example:
**   cd c:\tmp
*/
static void ficlChDir(FICL_VM *pVM) {
	FICL_STRING *pFS = (FICL_STRING*) pVM->pad;
	vmGetString(pVM, pFS, '\n');
	if (0 < pFS->count && _chdir(pFS->text)) {
		vmTextOut(pVM, "Error: path not found", 1);
		vmThrow(pVM, VM_QUIT);
	}
	return;
}

/* Here's the corresponding ficlBuild call:
** ficlBuild("cd", ficlChDir, FW_DEFAULT);
*/
\end{lstlisting}
To write a new FICL word in FORTH, follow the examples in
\textit{softcore.c}: embed the source in a string constant, and feed the
string to ficlExec once you have a virtual machine created. You'll also
find some examples of words coded in a mixture of C and FORTH in
\textit{words.c} (see 'evaluate' for example). Because ficlExec calls
can be nested, you can invoke ficlExec from within a function that
implements a word and feed it a string argument, effectively mixing the
two languages.

\section{Using FICL: An example}
Here's a quick example of FICL at work. Suppose we have a simple target
board that has a block of 8 LEDs, 8 DIP switches, an 8-bit analog to
digital converter, and an 8-bit DAC. (Conveniently, this is what FICL
simulates.) Let's start by setting up an address map:
\begin{lstlisting}[frame=single]
hex
10000 constant leds
10002 constant switches
20000 constant adc
20002 constant dac
\end{lstlisting}
What happened? The first line tells FICL that we're going to be writing
numbers in hexadecimal. The next four lines set up named constants for
the registers we want to use. When we invoke one of these constants (by
typing its name), it pushes its value.
\begin{lstlisting}[frame=single]
: !oreg ( value -- ) leds c! ;
: @adc  ( -- value ) adc c@ ;
: !dac  ( value -- ) dac c! ;
: @ireg ( -- value ) switches c@ ;
\end{lstlisting}
A constant pushes its value when invoked, and a variable pushes its
address. The FICL word @ (the at-sign) fetches the contents of an
address and puts the 32-bit value on the stack. FICL also has c@ to
fetch a byte, and w@ to fetch 16 bits. Likewise, !, w! and c! store
a value at an address. The syntax is ( value address -- ), meaning that
the operations consume a value and an address from the stack. The lines
above create FICL words that hide the width of the registers they use by
wrapping the fetch or store operation.
\begin{lstlisting}[frame=single]
variable led-shadow 0 led-shadow !
: !leds ( value -- ) dup !oreg led-shadow ! ;
0 !leds
\end{lstlisting}
The first line above creates a variable to track the LED state. (Real
hardware engineers often find it too expensive or too bothersome to add
a readback capability to digital output registers.) The word on the
second line writes the LED register and updates the shadow variable.
The last line forces the LEDs off.
\begin{lstlisting}[frame=single]
: toggle-led ( led -- )
	1 swap lshift    \ make a bitmak for the LED
	led-shadow @ xor \ toggle the bit in the shadow reg
	!oreg ;          \ now update the LED and shadow
\end{lstlisting}
This word toggles a LED by index (0..7). lshift is equivalent to C's
<< operator.
\begin{lstlisting}[frame=single]
: adc-loop
	begin
		@adc dup !dac 100 msec
	255 = until ;
\end{lstlisting}
The above word loops writing the ADC value back to the DAC until the
ADC value reaches 255. After each ADC sample, there is a 100  millisecond
pause.

There are extremely simple examples, but they give a feel for the accretive
process one uses to bring up hardware with FICL.

\section{FICL Quick Start}
\begin{tabular}{| l | l |}
	\hline
	.s & Displays the contents of the stack non-destructively \\ \hline
	. & Pops the top of the stack and displays it \\ \hline
	hex & Set number base to hex (decimal is the default) \\ \hline
	decimal & Set number base to decimal \\ \hline
	\textbf{Arithmetic} \\ \hline
	+ - * ( a b -- c) & Pop a and b off the stack, perform the operation, and push the result. Example\newline
	3 2 + ( leaves 5 on the stack ) \\ \hline
	negate ( x -- -x ) & Change the sign of the value on top of the stack.
	\textbf{Stack operations} \\ \hline
	dup ( x -- x x ) & copy the top of the stack \\ \hline
	swap (x y -- y x ) & swap the top two cells \\ \hline
	\textbf{Logic} \\ \hline
	true ( -- -1 ) & \\ \hline
	false ( -- 0 ) & False is always zero in FORTH, and true can be
	defined as false invert (all bits reversed). Used for logical
	and bitwise operations. Logical operations view any non-zero
	value as true, as... \\ \hline
	and or xor ( x y -- z ) & Perform bitwise operations on two
	arguments and push the result. \\ \hline
	invert ( x -- ~x ) & One's complement the top of the stack. \\ \hline
	< > <> = ( x y -- flag ) & Perform comparison and push the true or false \\ \hline
	\textbf{Fetch and Store} \\ hline
	@ w@ c@ ( address -- value ) & Fetch 32, 16, 8 bits respectively
	from address and leave the result on the stack. Values padded to
	32 bits \\ \hline
	! w! c! ( value -- address ) & Store 32, 16 or 8 bits respectively
	of value at address. \\ \hline
	\textbf{Creating new words} \\ \hline
	constant 
\end{tabular}

%!TEX encoding = UTF-8 Unicode
\chapter{Upgrading}
FICL 4.0 is smaller, faster, and more capable than any previous version.
For more information on why FICL 4.0 is so gosh-darned swell, see the
What's New In FICL 4.0? section of the overview.

Since the FICL API has changed so dramatically, you can't just drop the
new FICL source. You have two basic choices: use the
FICL\_WANT\_COMPATIBILITY support, and switching to the new API.

Note that using either of these choices requires that you recompile your
application. You cannot build FICL 4 into a shared library or DLL and
use it with an application expecting FICL 3.0. Stated another way: FICL
4 is source compatible but not binary compatible with FICL 3.


\section{Using FICL\textunderscore WANT\textunderscore COMPATIBILITY}
If you want to get FICL 4.0 up and running in your project as quickly
as possible, FICL\_WANT\_COMPATIBILITY is what you'll want to use.
There are two easy steps, one of which you might be able to skip:
\begin{enumerate}[noitemsep]
	\item Set the C preprocessor constant FICL\_WANT\_COMPATIBILITY
	to 1. The best way is by adding the following line to
	\textit{ficllocal.h}:
\begin{lstlisting}[frame=single]
#define FICL\_WANT\_COMPATIBILITY (1)
\end{lstlisting}

	\item If you use a custom ficlTextOut() function, you'll have to
	rename it, and explicitly specify it to FICL. Renaming it is
	necessary, because the FICL compatibility layer also provides
	one for code that called ficlTextOut() directly (instead of
	calling vmTextOut() as it should have). We recommend renaming
	your function to ficlTextOutLocal(), as we have have provided a
	prototype for this function for you in
	\textit{ficlcompatibility.h}. This will save you the trouble of
	defining your own prototype, ensuring you get correct name
	decoration / linkage, etc.

	There are two methods you can use to specify your ficlTextOut() function:
	\begin{enumerate}[noitemsep]
		\item Specify it in the FICL\_INIT\_INFO structure
		passed in to ficlInitSystem(). This is the preferred
		method, as it ensures you will see the results of FICL's
		initialization code, and it will be automatically passed
		in to every newly created VM.

		\item Set it explicitly in every VM by calling
		vmSetTextOut() and passing it in.
	\end{enumerate}
	\textbf{Note}: Any other method, such as setting it by hand in
	the FICL\_SYSTEM or FICL\_VM structures, will not work. There
	is a special compatibility layer for old-style OUTFUNC
	functions, but it is only invoked properly when you use one of
	the two methods mentioned above.
\end{enumerate}
This should be sufficient for you to recompile-and-go with FICL 4. If
it's not, please let us know, preferably including a suggested solution
to the problem.


\section{Using The New API}
Since most (all?) of the external symbols have changed names since the
3.0 series, here is a quick guide to get you started on renaming
everything. This is by no means an exhaustive list; this is meant to
guide you towards figuring out what the new name should be. (After all,
part of the point of this massive renaming was to make all the external
symbols consistent.)


\subsection{Types}
Every external type has been renamed. They all begin with the word ficl,
and they use mixed case (instead of all upper-case, which is now
reserved for macros). Also, the confusingly-named string objects have
been renamed: FICL\_STRING is now ficlCountedString, as it represents a
"counted string" in the language, and the more commonly-used STRINGINFO
is now simply ficlString.

\begin{tabular}{| l | l |}
	\hline
	\textbf{Old name} & \textbf{New name} \\ \hline
	FICL\_SYSTEM & ficlSystem \\ \hline
	FICL\_VM & ficlVm \\ \hline
	FICL\_SYSTEM\_INFO & ficlSystemInformation \\ \hline
	FICL\_WORD & ficlWord \\ \hline
	IPTYPE & ficlIp \\ \hline
	FICL\_CODE & ficlPrimitive \\ \hline
	OUTFUNC & ficlOutputFunction \\ \hline
	FICL\_DICTIONARY & ficlDictionary \\ \hline
	FICL\_STACK & ficlStack \\ \hline
	STRINGINFO & ficlString \\ \hline
	FICL\_STRING & ficlCountedString \\
	\hline
\end{tabular}


\subsection{Structure Members}
In addition, many structure names have changed. To help ease the
heartache, we've also added some accessor macros. So, in case they
change in the future, your code might still compile (hooray!).

\begin{tabular}{| l | l | l |}
	\hline
	\textbf{Old name} & \textbf{New name} & \textbf{Accessor} \\ \hline
	pExtend & context & ficlVmGetContext(), ficlSystemGetContext() \\ \hline
	pStack & dataStack & ficlVmGetDataStack() \\ \hline
	fStack & floatStack & ficlVmGetFloatStack() \\ \hline
	rStack & returnStack & ficlVmGetReturnStack() \\
	\hline
\end{tabular}


\subsection{Callback Functions}
Text output callbacks have changed in two major ways:
\begin{itemize}[noitemsep]
	\item They no longer take a VM pointer; they now take a
	ficlCallback structure. This allows output to be printed before
	a VM is defined, or in circumstances where a VM may not be
	defined (such as an assertion failure in a ficlSystem...()
	function).

	\item They no longer take a flag indicating whether or not to
	add a "newline". Instead, the function must output a newline
	whenever it encounters a \\n character in the text.
\end{itemize}
If you don't want to rewrite your output function yet, you can "thunk"
the new-style call to the old-style. Just pass in
ficlOldnamesCallbackTextOut as the name of the output function for the
system and VM, and then set the thunkedTextout member of the ficlSystem
or ficlVm to your old-style text output function.


\subsection{Renamed Macros}
\begin{tabular}{| l | l |}
	\hline
	\textbf{Old name} & \textbf{New name} \\ \hline
	PUSHPTR(p) & ficlStackPushPointer(vm-\textgreater dataStack, p) \\ \hline
	POPUNS() & ficlStackPopUnsigned(vm-\textgreater dataStack) \\ \hline
	GETTOP() & ficlStackGetTop(vm-\textgreater dataStack) \\ \hline
	FW\_IMMEDIATE & FICL\_WORD\_IMMEDIATE \\ \hline
	FW\_COMPILE & FICL\_WORD\_COMPILE\_ONLY \\ \hline
	VM\_INNEREXIT & FICL\_VM\_STATUS\_INNER\_EXIT \\ \hline
	VM\_OUTOFTEXT & FICL\_VM\_STATUS\_OUT\_OF\_TEXT \\ \hline
	VM\_RESTART & FICL\_VM\_RESTART \\
	\hline
\end{tabular}


\subsection{ficllocal.h}
One more note about macros. FICL now ships with a standard place for you
to tweak the FICL compile-time preprocessor switches such as
FICL\_WANT\_COMPATIBILITY and FICL\_WANT\_FLOAT. It's a file called
\textit{ficllocal.h}, and we guarantee that it will always ship empty
(or with only comments). We suggest that you put all your local changes
there, rather than editing \textit{ficl.h} or editing the Makefile.
That should make it much easier to integrate future FICL releases into
your product — all you need do is preserve your tweaked copy of
\textit{ficllocal.h} and replace the rest.


\subsection{Renamed Functions}
Every function that deals primarily with a particular structure is now
named after that structure. For instance, any function that takes a
ficlSystem as its first argument is named ficlSystemSomething(). Any
function that takes a ficlVm as its first argument is named
ficlVmSomething(). And so on.

Also, functions that create a new object are always called Create
(not Alloc, Allot, Init, or New). Functions that create a new object
are always called Destroy (not Free, Term, or Delete).

\begin{tabular}{| l | l |}
	\hline
	\textbf{Old name} & \textbf{New name} \\ \hline
	ficlInitSystem() & ficlSystemCreate() \\ \hline
	ficlTermSystem() & ficlSystemDestroy() \\ \hline
	ficlNewVM() & ficlSystemCreateVm() \\ \hline
	ficlFreeVM() & ficlVmDestroy() \\ \hline
	dictCreate() & ficlDictionaryCreate() \\ \hline
	dictDelete() & ficlDictionaryDestroy() \\ \hline
\end{tabular}
All functions exported by FICL now start with the word ficl. This is a
feature, as it means the FICL project will no longer pollute your
namespace.

\begin{tabular}{| l | l |}
	\hline
	\textbf{Old name} & \textbf{New name} \\ \hline
	PUSHPTR(p) & ficlStackPushPointer(vm-\textgreater dataStack, p) \\ \hline
	POPUNS() & ficlStackPopUnsigned(vm-\textgreater dataStack) \\ \hline
	GETTOP() & ficlStackGetTop(vm-\textgreater dataStack) \\ \hline
	ltoa() & ficlLtoa() \\ \hline
	strincmp() & ficlStrincomp() \\ \hline
\end{tabular}


\subsection{Removed Functions}
A few entry points have simply been removed. For instance, functions
specifically managing a system's ENVIRONMENT settings have been removed,
in favor of managing the system's environment dictionary directly:

\begin{tabular}{| l | l |}
	\hline
	\textbf{Old name} & \textbf{New name} \\ \hline
	ficlSystemSetEnvironment(system) & ficlDictionarySetConstant(ficlSystemGetEnvironment(system), ...) \\ \hline
	ficlSystemSet2Environment(system) & ficlDictionarySet2Constant(ficlSystemGetEnvironment(system), ...) \\ \hline
\end{tabular}
In a similar vein, ficlSystemBuild() has been removed in favor of using
ficlDictionarySetPrimitive() directly:

\begin{tabular}{| l | l |}
	\hline
	\textbf{Old name} & \textbf{New name} \\ \hline
	ficlSystemBuild(system, ...) & ficlDictionarySetPrimitive(ficlSystemGetDictionary(system), ...) \\ \hline
\end{tabular}
Finally, there is no exact replacement for ficlExec(). 99\% of the code
that called ficlExec() never bothered to manage SOURCE-ID properly. If
you were calling ficlExec(), and you weren't changing SOURCE-ID
(or vm-\textgreater sourceId) to match, you should replace those calls
with ficlVmEvaluate(), which will manage SOURCE-ID for you.

There is a function that takes the place of ficlExec() which doesn't
change SOURCE-ID: ficlVmExecuteString(). However, instead of taking a
straight C string (a char *), it takes a ficlString * as its code
argument. (This is to discourage its use.)


\section{Internal Changes}
\textbf{Note}: none of these changes should affect you. If they do,
there's probably a problem somewhere. Either FICL's API doesn't abstract
away something enough, or you are approaching a problem the wrong way.
Food for thought.

There's only one internal change worth noting here. The top value on a
FICL stack used to be at (to use the modern structure names)
stack-\textgreater top[-1]. It is now at stack-\textgreater top[0]. In
other words, the "stack top" pointer used to point past the top element;
it now points at the top element. (Pointing at the top element is not
only less confusing, it is also faster.)

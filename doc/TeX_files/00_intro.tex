%!TEX encoding = UTF-8 Unicode
\chapter{Intro}
\section{What is Ficl}
Ficl is a programming language interpreter designed to be embedded
into other systems as a command, macro and development
prototyping language.

Ficl is an acronym for "Forth Inspired Command Language"
\section{Ficl features}
\subsection{Ficl is \textbf{easy to port}.}
\begin{itemize}[noitemsep]
	\item It typically takes under 2 hours to port to a new platform.
	\item Ficl is written in strict ANSI C.
	\item Ficl can run natively on 32- and 64-bit processors.
\end{itemize}
\subsection{Ficl has a \textbf{small memory footprint}.}
A fully featured Win32 console version takes less than 100K of
memory, and a minimal version is less than half of that.
\subsection{Ficl is \textbf{easy to integrate} into your program.}
Where most Forths view themselves as the center of the system
and expect the rest of the system to be coded in Forth, Ficl
acts as a component of your program.  It is easy to export code
written in C or ASM to Ficl (in the style of TCL), or to invoke Ficl
code from a compiled module.
\subsection{Ficl is \textbf{fast}.}
Thanks to its "switch-threaded" virtual machine design, Ficl 4 is faster
than ever—about 3x the speed of Ficl 3. Ficl also features blindingly
fast "just in time" compiling, removing the "compile" step from the
usual compile-debug-edit iterative debugging cycle.
\subsection{Ficl is a \textbf{complete and powerful programming language}.}
Ficl is an implementation of the FORTH language, a language providing a
wide range of standard programming language features:
\begin{itemize}[noitemsep]
	\item Integer and floating-point numbers, with a rich set of operators.
	\item Arrays.
	\item File I/O.
	\item Flow control (if/then/else and many looping structures).
	\item Subroutines with named arguments.
	\item Language extensibility.
	\item Powerful code pre-processing features.
\end{itemize}
\subsection{Ficl is \textbf{standards-compliant}.}
Ficl conforms to the 1994 ANSI Standard for FORTH (DPANS94). See ANS
Required Information for more detail.
\subsection{Ficl is \textbf{extensible}.}
Ficl is extensible both at compile-time and at run-time. You can add new
script functions, new native functions, even new control structures.
\subsection{Ficl adds \textbf{object oriented programming features}.}
Ficl's flexible OOP library can be used to wrap data structures or
classes of the host system without altering them. (And remember how we
said Ficl was extensible? Ficl's object-oriented programming extensions
are written in Ficl.)
\subsection{Ficl is \textbf{interactive}.}
Ficl can be used interactively, like most other FORTHs, Python, and
Smalltalk. You can inspect data, run commands, or even define new
commands, all on a running Ficl VM. Ficl also has a built-in script
debugger that allows you to step through Ficl code as it is executed.
\subsection{Ficl is \textbf{ROMable}.}
Ficl is designed to work in RAM based and ROM code / RAM data environments.
\subsection{Ficl is \textbf{safe for multithreaded programs}.}
Ficl is reentrant and thread-safe. After initialization, it does not
write to any global data.
\subsection{Ficl is \textbf{open-source and free}.}
The Ficl licence is a BSD-style license, requiring only that you
document that you are using Ficl. There are no licensing costs for using
Ficl.
\section{What's New In Ficl 4.0}
Ficl 4.0 is a major change for Ficl. Ficl 4.0 is smaller, faster, more
powerful, and easier to use than ever before. (Or your money back!)

Ficl 4.0 features a major engine rewrite. Previous versions of Ficl
stored compiled words as an array of pointers to data structure; Ficl
4.0 adds "instructions", and changes over to mostly using a
"switch-threaded" model. The result? Ficl 4.0 is approximately three
times as fast as Ficl 3.03.

Ficl 4.0 also adds the ability to store the "softcore" words as LZ77
compressed text. Decompression is so quick as to be nearly unmeasurable
(0.00384 seconds on a 750MHz AMD Duron-based machine). And even with the
runtime decompressor, the resulting Ficl executable is over 13k smaller!

Another new feature: Ficl 4.0 can take advantage of native support for
double-word math. If your platform supports it, set the preprocessor
symbol FICL\_HAVE\_NATIVE\_2INTEGER to 1, and create typedefs for
ficl2Integer and ficl2Unsigned.

Ficl 4.0 also features a retooled API, and a redesigned directory tree.
The API is now far more consistent. But for those of you who are
upgrading from Ficl 3.03 or before, you can enable API backwards
compatibility by turning on the compile-time flag FICL\_WANT\_COMPATIBILITY.

Ficl 4.0 also extends support every kind of local and global value
imaginable. Every values can individually be local or global, single-cell
or double-cell, and integer or floating-point. And TO always does the
right thing.

If you're using Ficl under Windows, you'll be happy to know that there's
a brand-new build process. The Ficl build process now builds Ficl as
\begin{itemize}[noitemsep]
	\item a static library (.LIB),
	\item a dynamic library (.DLL, with a .LIB import library), and
	\item a standalone executable (.EXE).
\end{itemize}
Furthermore, each of these targets can be built in Debug or Release,
Singlethreaded or Multithreaded, and optionally using the DLL version
of the C runtime library for Multithreaded builds. (And, plus, the
/objects/common nonsense is gone!)

Finally, Ficl 4.0 adds a contrib directory, a repository for
user-contributed code that isn't part of the standard FICL release. The
only package there right now is XClasses, a Python-based IDL that
generates the definition files for C++-based classes, the equivalent
FICL classes, and code to allow the FICL classes to call the C++ methods.
Using XClasses you can write your class once, and use it immediately
from both C++ and FICL.

\section{Getting FICL}
You can download FICL from the FICL download page at Sourceforge.

\section{Building FICL}
\subsection{Building using CMake}
\subsection{Building using Makefile}

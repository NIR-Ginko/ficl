%!TEX encoding = UTF-8 Unicode
\chapter{API}
\section{Quick FICL Programming Concepts Overview}
A FICL dictionary is equivalent to the FORTH "dictionary"; it is where
words are stored. A single dictionary has a single HERE pointer.

A FICL system information structure is used to change default values
used in initializing a FICL system.

A FICL system contains a single dictionary, and one or more virtual
machines.

A FICL stack is equivalent to a FORTH "stack". FICL has three stacks:
\begin{itemize}[noitemsep]
	\item The data stack, where integer arguments are stored.
	\item The return stack, where locals and return addresses for
		subroutine returns are stored.
	\item The float stack, where floating-point arguments are
		stored. (This stack is only enabled when
		FICL\_WANT\_FLOAT is nonzero.)
\end{itemize}
A FICL virtual machine (or VM) represents a single running instance of
the FICL interpreter. All virtual machines in a single Ficl system see
the same dictionary.


\section{Quick FICL Programming Tutorial}
Though FICL's API offers a great deal of flexibility, most programs
incorporating FICL simply use it as follows:
\begin{enumerate}[noitemsep]
	\item Create a single ficlSystem using ficlSystemCreate(NULL).
	\item Add native functions as necessary with
		ficlDictionarySetPrimitive().
	\item Add constants as necessary with
		ficlDictionarySetConstant().
	\item Create one (or more) virtual machine(s) with
		ficlSystemCreateVm().
	\item Add one or more scripted functions with ficlVmEvaluate().
	\item Execute code in a Ficl virtual machine, usually with
		ficlVmEvaluate(), but perhaps with ficlVmExecuteXT().
	\item At shutdown, call ficlSystemDestroy() on the single FICL
		system.
\end{enumerate}


\section{FICL Application Programming Interface}
The following is a partial listing of functions that interface your
system or program to FICL. For a complete listing, see ficl.h (which is
heavily commented). For a simple example, see main.c.

Note that as of Ficl 4, the API is internally consistent. Every external
entry point starts with the word \textit{ficl}, and the word after that
also corresponds with the first argument. For instance, a word that
operates on a ficlSystem* will be called ficlSystemSomething().
\begin{itemize}[noitemsep]
	\item void ficlSystemInformationInitialize(ficlSystemInformation
		*fsi)\newline
	Resets a ficlSystemInformation structure to all zeros. (Actually
	implemented as a macro.) Use this to initialize a
	ficlSystemInformation structure before initializing its members
	and passing it into ficlSystemCreate() (below).

	\item ficlSystem *ficlSystemCreate(ficlSystemInformation
		*fsi)\newline
	Initializes FICL's shared system data structures, and creates
	the dictionary allocating the specified number of cells from the
	heap (by a call to ficlMalloc()). If you pass in a NULL pointer,
	you will recieve a ficlSystem using the default sizes for the
	dictionary and stacks.

	\item void ficlSystemDestroy(ficlSystem *system)\newline
	Reclaims memory allocated for the Ficl system including all
	dictionaries and all virtual machines created by
	ficlSystemCreateVm(). Note that this will not automatically free
	the memory allocated by the FORTH memory allocation words
	(ALLOCATE and RESIZE).

	\item ficlWord *ficlDictionarySetPrimitive(ficlDictionary
		*dictionary, char *name, ficlCode code, ficlUnsigned8
		flags)\newline
	Adds a new word to the dictionary with the given name, code
	pointer, and flags.

	The flags parameter is a bitfield. The valid flags are:
	\begin{itemize}[noitemsep]
		\item FICL\_WORD\_IMMEDIATE
		\item FICL\_WORD\_COMPILE\_ONLY
		\item FICL\_WORD\_SMUDGED
		\item FICL\_WORD\_OBJECT
		\item FICL\_WORD\_INSTRUCTION
	\end{itemize}
	For more information on these flags, see \textit{ficl.h}.

	\item int ficlVmEvaluate(ficlVm *vm, char *text)\newline
	Pass the specified C string (zero-terminated) to the given
	virtual machine for evaluation. Returns various exception codes
	(VM\_XXXX in ficl.h) to indicate the reason for returning.
	Normal exit condition is VM\_OUTOFTEXT, indicating that the VM
	consumed the string successfully and is back for more. Calls to
	ficlVmEvaluate() can be nested, and the function itself is
	re-entrant, but note that a VM is static, so you have to take
	reasonable precautions (for example, use one VM per thread in a
	multithreaded system if you want multiple threads to be able to
	execute commands).

	\item int ficlVmExecuteXT(ficlVm *vm, ficlWord *pFW)\newline
	Same as ficlExec, but takes a pointer to a ficlWord instead of
	a string. Executes the word and returns after it has finished.
	If executing the word results in an exception, this function
	will re-throw the same code if it is nested under another
	ficlExec family function, or return the exception code directly
	if not. This function is useful if you need to execute the same
	word repeatedly — you save the dictionary search and outer
	interpreter overhead.

	\item void ficlFreeVM(ficlVm *vm)\newline
	Removes the VM in question from the system VM list and deletes
	the memory allocated to it. This is an optional call, since
	ficlTermSystem will do this cleanup for you. This function is
	handy if you're going to do a lot of dynamic creation of VMs.

	\item ficlVm *ficlNewVM(ficlSystem *system)\newline
	Create, initialize, and return a VM from the heap using
	ficlMalloc. Links the VM into the system VM list for later
	reclamation by ficlTermSystem.

	\item ficlWord *ficlSystemLookup(ficlSystem *system, char
		*name)\newline
	Returns the address of the specified word in the main
	dictionary. If no such word is found, it returns NULL. The
	address is also a valid execution token, and can be used in a
	call to ficlVmExecuteXT().

	\item ficlDictionary *ficlSystemGetDictionary(ficlSystem
		*system)\newline ficlDictionary
		*ficlVmGetDictionary(ficlVm *system)\newline
	Returns a pointer to the main system dictionary.

	\item ficlDictionary *ficlSystemGetEnvironment(ficlSystem
		*system)\newline
	Returns a pointer to the environment dictionary. This dictionary
	stores information that describes this implementation as
	required by the Standard.

	\item ficlDictionary *ficlSystemGetLocals(ficlSystem
		*system)\newline
	Returns a pointer to the locals dictionary. This function is
	defined only if FICL\_WANT\_LOCALS is non-zero (see
	\textit{ficl.h}). The locals dictionary is the symbol table for
	local variables.
\end{itemize}

\section{FICL Compile-Time Constants}
There are a lot of preprocessor constants you can set at compile-time
to modify FICL's runtime behavior. Some are required, such as telling
FICL whether or not the local platform supports double-width integers
(FICL\_PLATFORM\_HAS\_2INTEGER); some are optional, such as telling FICL
whether or not to use the extended set of "prefixes"
(FICL\_WANT\_EXTENDED\_PREFIXES).

The best way to find out more about these constants is to read
\textit{ficl.h} yourself. The settings that turn on or off FICL modules
are all starting with FICL\_WANT. The settings related to functionality
available on the current platform are all starting with FICL\_PLATFORM.


\section{ficllocal.h}
One more note about constants. FICL now ships with a standard place for
you to tweak the FICL compile-time preprocessor constants. It's a file
called \textit{ficllocal.h}, and we guarantee that it will always ship
empty (or with only comments). We suggest that you put all your local
changes there, rather than editing \textit{ficl.h} or editing the
Makefile. That should make it much easier to integrate future FICL
releases into your product — all you need do is preserve your tweaked
copy of \textit{ficllocal.h} and replace the rest.

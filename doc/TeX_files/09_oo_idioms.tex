%!TEX encoding = UTF-8 Unicode
\chapter{Object Oriented Idioms in C}
John Sadler
10 Aug 98
\section{Review of Key OO Characteristics}
\subsection{Object}
\textbf{"A package of information and descriptions of its manipulation"
[Robson]}

Objects separate interface from implementation, "what" is wanted (on
the outside) from "how" it is accomplished (on the inside). This idea
of hiding the data inside a package with a fixed set of allowed
manipulations is called encapsulation. Why? If the object always selects
how to perform a requested manipulation, you guarantee that the
procedure and the data it operates on always match.

A \textbf{message} denotes an operation that can be performed on an
\textbf{object}. The code that describes how to perform an operation on
a specific object type is called a \textbf{method}. From the outside,
objects receive messages. On the inside, these messages are mapped to
methods that perform appropriate actions for the specific kind of object.
An object’s \textbf{interface} is the set of messages to which it can
respond. For example, several object types may have a "dump" method to
cause the object to display its state. Each kind of object will need a
unique method to accomplish this. So the message-method idea separates
the interface from the implementation. The idea that different kinds of
objects might invoke different methods to respond to the same message is
called \textbf{polymorphism}. Methods can be bound to messages as soon
as the type of the object receiving the message is known (called
\textbf{early binding}), or this mapping can wait until run-time
(\textbf{late binding}).

Some languages (notably C++) make a syntactic distinction between early
and late binding of methods to messages (virtual functions are bound
late, while all others are bound early – at link time). Smalltalk makes
no such distinction. The C++ approach has potentially dangerous
consequences when you manipulate an object through a pointer to its
parent class. If a method of the superclass is virtual, you get late
binding to the appropriate function for the object’s class. On the other
hand, if that method is not virtual, you get early binding to the parent
class’s definition even if the child class has overridden it. Smalltalk
adheres rigorously to the idea that the object itself has sole ownership
of the mapping from methods to messages. 


\subsection{Class}
\textbf{"A description of one or more similar objects" [Robson]}

A specific object described by a particular class is called an
\textbf{instance} of the class.

(Example: Dog is a class; Poodle is a subclass of Dog; FiFi is an
object, an instance of Poodle).

A class is a kind of object that describes the behaviors (methods) of
its instances, and whose methods provide for creation, initialization,
and destruction of an instance. All instances of a particular class use
the same method to respond to a given message. Classes may define other
members that are shared by all instances of the class. These are called
class variables. (In C++, these would be static members.) 


\subsection{Inheritance}
A means for creating a new object or class using an existing one as a
starting place and defining only what changes. C++ only supports
class-based inheritance, but some systems also allow objects to inherit
directly from other objects. At minimum, inheritance requires that the
child class override its parent’s name. In addition, a child class can:
\begin{itemize}[noitemsep]
	\item add instance variables
	\item add class variables
	\item define methods for new messages
	\item provide methods for (\textbf{override}) messages already
	handled by the parent class
\end{itemize}


\section{What C++ Does for you…}
\subsection{Name mangling – separation of name spaces}
This mechanism – decorating symbol names with extra characters that
uniquely identify their class and prototype – is the traditional C++
method for operator overloading, method overloading, and namespace
separation. C provides a much smaller set of namespaces than C++
requires, so this strategy allowed C++ to be implemented as a
preprocessor for C compilers originally.


\subsection{Rigorous type checking}
Because of the function prototype requirement and name mangling, a C++
compiler can provide strict type checking for method invocations.


\subsection{Automatic lifetime control}
Guarantees constructor call upon creation and destructor call upon
deletion.


\subsection{Multiple Storage classes (same as C)}
Automatic, static, dynamically allocated


\subsection{Typed dynamic memory management}


\subsection{Default constructor, destructor, copy constructor, and assignment operator}


\subsection{Explicit early and late binding support}


\subsection{And more…}


\section{What other OO languages do (or don’t do)}
Just to make sure we don’t get caught in a C++ centered view of the
world, here are some ways other OO systems differ.


\subsection{Run-time type identification}
This is a Big Deal in C++, but it’s relatively trivial in an interpreted
language to know the type of a reference at run-time.


\subsection{Metaclasses}
Classes are objects, too (Smalltalk, Java?)


\subsection{Garbage collection}
Java and Smalltalk both manage memory for you, freeing objects when they
go out of scope or are no longer referenced anywhere.


\subsection{Single Inheritance only}
Java, Smalltalk 80


\subsection{Everything is an object}
Smalltalk


\subsection{Operator overloading}
Smalltalk makes no distinction between operators and other kinds of
messages. The message syntax is flexible enough that you can define
operators in the same way as any other kind of message.


\subsection{Visibility control}
Smalltalk 80 does not appear to provide options for visibility control
(based on my quick survey). Instance variables are always private,
methods are always public as far as I can tell.


\subsection{Pointers}
Not in Smalltalk, Java: Both languages deal with objects through
implicit references. It is still possible to create data structures,
but the language hides much of the memory management work.


\subsection{No Casting}
As far as I can tell, smalltalk has no equivalent of a C/C++ cast.


\subsection{Late binding}
Smalltalk makes no syntactic distinction between late and early bound
methods (unlike C++ "virtual" methods) 


\section{OO-C framework options}
\begin{itemize}[noitemsep]
	\item Covered: objects (encapsulation, explicit construct and
	destruct), classes, inheritance, polymorphism 

	\item Not covered: multiple inheritance, automatic
	initialization / destruction. 
\end{itemize}


\subsection{Strategy 1: message maps and aggregation}
Class: a struct with a pointer to a method table (message map).
Contructor and destuctor are really initializer and destructor, and are
invoked manually at beginning and end of lifetime.

Messages and methods: mapping table (first cut: use macros like MFC to
create a mapping between messages and methods). Alternative: message map
can be built at run-time (hash, tree, linked list) – method resolution
may be slower.

Inheritance: aggregate the parent struct (recursively) at the beginning
of the derived one, link the child method table to the parent and search
recursively to resolve messages to methods

Pros and cons:
+ flexible and minimal manual steps required
+ relatively simple – no need to write a preprocessor!
- all methods are late bound, run-time penalty defeats compile time type
checking, may require a single prototype for all methods


\subsection{Strategy 2: manual name mangling}
This is how the Samek article handles encapsulation


\subsection{Strategy 3: preprocessor}
This is how C++ started out.


\subsection{References}
\begin{itemize}[noitemsep]
	\item David Robson, Object Oriented Software Systems. Byte,
	August 1981

	\item Miro Samek, Portable Inheritance and Polymorphism in C.
	Embedded Systems Programming, December 1997
\end{itemize}

%!TEX encoding = UTF-8 Unicode
\chapter{Debugger}
Ficl includes a simple step debugger for colon definitions and DOES\textgreater words.
\section{Using The Ficl Debugger}
To debug a word, set up the stack with any parameters the word requires, then execute:

\textbf{DEBUG your-word-name-here}

If the word is unnamed, or all you have is an execution token, you can
instead use DEBUG-XT

The debugger invokes SEE on the word which prints a crude source
listing. It then stops at the first instruction of the definition.
There are six (case insensitive) commands you can use from here onwards:
\begin{itemize}[noitemsep]
	\item \textbf{I} (step \textbf{I}n)\newline
	If the next instruction is a colon defintion or does\textgreater
	word, steps into that word's code. If the word is a primitive,
	simply executes the word.

	\item \textbf{O} (step \textbf{O}ver)\newline
	Executes the next instruction in its entirety.

	\item \textbf{G} (\textbf{G}o)\newline
	Run the word to completion and exit the debugger.

	\item \textbf{L} (\textbf{L}ist)\newline
	Lists the source code of the word presently being stepped.

	\item \textbf{Q} (\textbf{Q}uit)\newline
	Abort the word and exit the debugger, clearing the stacks.

	\item \textbf{X} (e\textbf{X}ecute)\newline
	Interpret the remainder of the line as FICL words. Any change
	they make to the stacks will be preserved when the debugged
	word continues execution. Any errors will abort the debug
	session and reset the VM. Usage example:
	\begin{lstlisting}[frame=single]
X DROP 3 \ change top argument on stack to 3
	\end{lstlisting}
\end{itemize}
Any other character will print a list of available debugger commands.


\section{The ON-STEP Event}
If there is a defined word named ON-STEP when the debugger starts, that
word will be executed before every step. Its intended use is to display
the stacks and any other VM state you find interesting. The default
ON-STEP is:
\begin{lstlisting}[frame=single]
: ON-STEP  ." S: " .S-SIMPLE CR ;
\end{lstlisting}
If you redefine ON-STEP, we recommend you ensure the word has no
side-effects (for instance, adding or removing values from any stack).


\section{Other Useful Words For Debugging And ON-STEP}
\begin{itemize}
	\item .ENV ( -- )\newline
	Prints all environment settings non-destructively.

	\item .S ( -- )\newline
	Prints the parameter stack non-destructively in a verbose
	format.

	\item .S-SIMPLE ( -- )\newline
	Prints the parameter stack non-destructively in a simple
	single-line format.

	\item F.S ( -- )\newline
	Prints the float stack non-destructively (only available if
	FICL\_WANT\_FLOAT is enabled).

	\item R.S ( -- )\newline
	Prints a represention of the state of the return stack
	non-destructively.
\end{itemize}


\section{Debugger Internals}
The debugger words are mostly located in source file \textit{tools.c}.
There are supporting words (\textit{DEBUG} and \textit{ON-STEP}) in
\textit{softcore/softcore.fr} as well. There are two main words that
make the debugger go: \textit{debug-xt} and \textit{step-break}.
\textit{debug-xt} takes the execution token of a word to debug (as
returned by \textit{'} for example), checks to see if it is debuggable
(not a primitive), sets a breakpoint at its first instruction, and runs
\textit{SEE} on it. To set a breakpoint, \textit{debug-xt} replaces the
instruction at the breakpoint with the execution token of
\textit{step-break}, and stores the original instruction and its address
in a static breakpoint record. To clear the breakpoint,
\textit{step-break} simply replaces the original instruction and adjusts
the target virtual machine's instruction pointer to run it.

\textit{step-break} is responsible for processing debugger commands and
setting breakpoints at subsequent instructions.


\section{Future Enhancements}
\begin{itemize}[noitemsep]
	\item The debugger needs to exit automatically when it
	encounters the end of the word it was asked to debug. (Perhaps
	this could be a special kind of breakpoint?)

	\item Add user-set breakpoints.
	\item Add "step out" command.
\end{itemize}

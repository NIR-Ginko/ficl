%!TEX encoding = UTF-8 Unicode
\chapter{Local variables in Ficl}
\section{An Overview And A History}
Named, locally scoped variables came late to FORTH. Purists feel that
experienced FORTH programmers can (and should) write supportable code
using only anonymous stack variables and good factoring, and they
complain that novices use global variables too frequently. But local
variables cost little in terms of code size and execution speed, and
are very convenient for OO programming (where stack effects are more
complex).

FICL provides excellent support for local variables, and the purists be
damned — we use 'em all the time.

Local variables can only be declared inside a definition, and are only
visible in that definition. Please refer to the ANS standard for FORTH
for more general information on local variables.


\section{John-Hopkins Forth Argument Syntax}
ANS Forth does not specify a complete local variable facility. Instead,
it defines a foundation upon which to build one. FICL comes with an
adaptation of the Johns-Hopkins local variable syntax, as developed by
John Hayes et al. However, FICL extends this syntax with support for
double-cell and floating-point numbers.

Here's the basic syntax of a JH-local variable declaration:

{ arguments | locals -- ignored }

(For experienced FORTH programmers: the declaration is designed to look
like a stack comment, but it uses curly braces instead of parentheses.)
Each section must list zero or more legal FICL word names; comments and
preprocessing are not allowed here. Here's what each section denotes:
\begin{itemize}[noitemsep]
	\item The arguments section lists local variables which are
	initialized from the stack when the word executes. Each argument
	is set to the top value of the stack, starting at the rightmost
	argument name and moving left. You can have zero or more
	arguments.

	\item The \textit{locals} section lists local variables which
	are set to zero when the word executes. You can have zero or
	more locals.

	\item Any characters between -- and \} are treated as a
	comment, and ignored.
\end{itemize}

(The | and -- sections are optional, but they must appear in the order
shown if they appear at all.)


\section{Argument Types}
Every time you specify a local variable (in either the arguments or the
locals section), you can also specify the type of the local variable.
By default, a local variable is a single-cell integer; you can specify
that the local be a double-cell integer, and/or a floating-point number.

To specify the type of a local, specify one or more of the following
single-character specifiers, followed by a colon (:).
\begin{itemize}[noitemsep]
	\item \textbf{1} single-cell
	\item \textbf{2} double-cell
	\item \textbf{d} double-cell
	\item \textbf{f} floating-point (use floating stack)
	\item \textbf{i} integer (use data stack)
	\item \textbf{s} single-cell
\end{itemize}
For instance, the argument f2:foo would specify a double-width
floating-point number.

The type specifiers are read right-to left, and when two specifiers
conflict, the rightmost one takes priority. So 2is1f2:foo would still
specifiy a double-width floating-point number.

Note that this syntax only works for FICL's JH-locals. Locals defined in
some other way (say, with the FORTH standard word LOCALS|) will ignore
this syntax, and the entire string will be used as the name of the local
(type and all).


\section{A Simple Example}
\begin{lstlisting}[frame=single]
: DEMONSTRATE-JH-LOCALS { c b a  f:float -- a+b f:float*2 }
  a b +
  2.0e float f* ;
\end{lstlisting}

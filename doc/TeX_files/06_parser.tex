\chapter{FICL parse steps}
\section{Parse steps}
Unlike every other FORTH we know of, FICL features an extensible parser
chain. The FICL parser is not a monolithic function; instead, it is
comprised of a simple tokenizer and a series of parse steps. A parse
step is a step in the parser chain that handles a particular kind of
token, acting on the token as appropriate. Example parse steps, in terms
of traditional FORTH lore, would be the "number runner" and the "colon
compiler".

The FICL parser works like this:
\begin{enumerate}[noitemsep]
	\item Read in a new token (string of text with no internal
	whitespace).

	\item For each parse step in the chain, call the parse step,
	passing in the token. If the parse step returns FICL\_TRUE, that
	parse step must have handled the token appropriately; move on to
	the next token.

	\item If the parser tries all the parse steps and none of them
	return FICL\_TRUE, the token is illegal — print an error and
	reset the virtual machine.
\end{enumerate}
Parse steps can be written as native functions, or as FICL script
functions. New parse steps can be appended to the chain at any time.


\section{The Default FICL Parse Chain}
These is the default FICL parser chain, shown in order.
\begin{itemize}[noitemsep]
	\item ?word\newline
	If compiling and local variable support is enabled, attempt to
	find the token in the local variable dictionary. If found,
	execute the token's compilation semantics and return FICL\_TRUE.

	Attempt to find the token in the system dictionary. If found,
	execute the token's semantics (may be different when compiling
	than when interpreting) and return FICL\_TRUE.

	\item ?prefix\newline
	This parse step is only active if prefix support is enabled,
	setting FICL\_WANT\_PREFIX in \textit{ficl.h} to a non-zero
	value. Attempt to match the beginning of the token to the list
	of known prefixes. If there's a match, execute the associated
	prefix method and return FICL\_TRUE.

	\item ?number\newline
	Attempt to convert the token to a number in the present BASE. If
	successful, push the value onto the stack if interpreting,
	otherwise compile it, then return FICL\_TRUE.

	\item ?float\newline
	This parse step is only active if floating-point number support
	is enabled, setting FICL\_WANT\_FLOAT in \textit{ficl.h} to a
	non-zero value. Attempt to convert the token to a floating-point
	number. If successful, push the value onto the floating-point
	stack if interpreting, otherwise compile it, then return FICL\_TRUE.
\end{itemize}


\section{Adding A Parse Step From Within FICL}
You can add a parse step in two ways. The first is to write a FICL word
that has the correct stack signature for a parse step:
\begin{lstlisting}[frame=single]
MY-PARSE-STEP   ( c-addr u -- x*i flag )
\end{lstlisting}
where c-addr u are the address and length of the incoming token, and
flag is FICL\_TRUE if the parse step processed the token and FICL\_FALSE
otherwise.

Install the parse step using add-parse-step. A trivial example:
\begin{lstlisting}[frame=single]
: ?silly   ( c-addr u -- flag )
." Oh no! Not another  " type cr  true ;
' ?silly add-parse-step
parse-order
\end{lstlisting}


\section{Adding A Native Parse Step}
The other way to add a parse step is to write it in C and add it into
the parse chain with the following function:
\begin{lstlisting}[frame=single]
void ficlSystemAddPrimitiveParseStep(ficlSystem *system, char *name, ficlParseStep step);
\end{lstlisting}
name is the display name of the parse step in the parse chain (as
displayed by the FICL word PARSE-ORDER). Step is a pointer to the code
for the parse step itself, and must match the following declaration:
\begin{lstlisting}[frame=single]
typedef int (*ficlParseStep)(ficlVm *vm, ficlString s);
\end{lstlisting}
When a native parse step is run, si points to the incoming token. The
parse step must return FICL\_TRUE if it succeeds in handling the token,
and FICL\_FALSE otherwise. See ficlVmParseNumber() in
\textit{src/system.c} for an example.


\section{Prefixes}
What's a prefix, anyway? A prefix (contributed by Larry Hastings) is a
token that's recognized as the beginning of another token. Its presence
modifies the semantics of the rest of the token. An example is 0x, which
causes digits following it to be converted to hex regardless of the
current value of BASE.

Caveat: Prefixes are matched in sequence, so the more of them there are,
the slower the interpreter gets. On the other hand, because the prefix
parse step occurs immediately after the dictionary lookup step, if you
have a prefix for a particular purpose, using it may save time since it
stops the parse process. Also, the FICL interpreter is wonderfully fast,
and most interpretation only happens once, so it's likely you won't
notice any change in interpreter speed even if you make heavy use of
prefixes.

Each prefix is a FICL word stored in a special wordlist called
\textless PREFIXES\textgreater . When the prefix parse step (?prefix,
implemented in C as ficlVmParsePrefix()) is executed, it searches each
word in \textless PREFIXES\textgreater  in turn, comparing it with the
initial characters of the incoming token. If a prefix matches, the parse
step returns the remainder of the token to the input stream and executes
the code associated with the prefix. This code can be anything you like,
but it would typically do something with the remainder of the token. If
the prefix code does not consume the rest of the token, it will go
through the parse process again (which may be what you want).

Prefixes are defined in \textit{src/prefix.c} and in
\textit{softcore/prefix.fr}. The best way to add prefixes is by defining
them in your own code, bracketed with the special words START-PREFIXES
and END-PREFIXES. For example, the following code would make .( a prefix.
\begin{lstlisting}[frame=single]
start-prefixes
: .(  .( ;
end-prefixes
\end{lstlisting}
The compile-time constant FICL\_EXTENDED\_PREFIX controls the inclusion
of several additional prefixes. This is turned off in the default build,
since several of these prefixes alter standard behavior, but you might
like them.


\section{Notes}
\begin{itemize}[noitemsep]
	\item Prefixes and parser extensions are non-standard. However,
	with the exception of prefix support, FICL's default parse order
	follows the standard. Inserting parse steps in some other order
	will almost certainly break standard behavior.

	\item The number of parse steps that can be added to the system
	is limited by the value of FICL\_MAX\_PARSE\_STEPS (defined in
	\textit{src/sysdep.h}). The default maximum number is 8.

	\item The compile-time constant FICL\_EXTENDED\_PREFIX controls
	the inclusion of several additional prefixes. This is turned off
	in the default build, since several of these prefixes alter
	standard behavior, but you might like them.
\end{itemize}


\section{Parser Glossary}
\begin{itemize}[noitemsep]
	\item PARSE-ORDER ( -- )\newline
	Prints the list of parse steps, in the order in which they are
	called.

	\item ADD-PARSE-STEP ( xt -- )\newline
	Appends a parse step to the parse chain. xt is the address
	(execution token) of a FICL word to use as the parse step. The
	word must be a legal FICL parse step (see above).

	\item SHOW-PREFIXES ( -- )\newline
	Prints the list of all prefixes. Each prefix is a FICL word that
	is executed if its name is found at the beginning of a token.

	\item START-PREFIXES ( -- )\newline
	Declares the beginning of a series of prefix definitions. Should
	be followed, eventually, by END-PREFIXES. (All START-PREFIXES
	does is tell the FICL virtual machine to compile into the
	\textless PREFIXES\textgreater  wordlist.)

	\item END-PREFIXES ( -- )\newline
	Declares the end of a series of prefix definitions. Should only
	be used after calling START-PREFIXES. (All END-PREFIXES does is
	tell the FICL virtual machine to switch back to the wordlist
	that was in use before START-PREFIXES was called.)
\end{itemize}

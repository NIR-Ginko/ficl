%!TEX encoding = UTF-8 Unicode
\chapter{ANS Required Information}
The following documentation is necessary to for Ficl to comply with the
DPANS94 standard. It describes what areas of the standard Ficl
implements, what areas it does not, and how it behaves in areas
undefined by the standard.


\section{ANS Forth System}
\begin{itemize}[noitemsep]
	\item Providing names from the Core Extensions word set
	\item Providing names from the Double-Number word set
	\item Providing the Exception word set
	\item Providing the Exception Extensions word set
	\item Providing the File-Access word set
	\item Providing the File-Access Extensions word set
	\item Providing names from the Floating-Point word set
	\item Providing the Locals word set
	\item Providing the Locals Extensions word set
	\item Providing the Memory Allocation word set
	\item Providing the Programming-Tools word set
	\item Providing names from the Programming-Tools Extensions word set
	\item Providing the Search-Order word set
	\item Providing the Search-Order Extensions word set
	\item Providing names from the String Extensions word set
\end{itemize}


\section{Implementation-defined Options}
The implementation-defined items in the following list represent
characteristics and choices left to the discretion of the implementor,
provided that the requirements of the Standard are met. A system shall
document the values for, or behaviors of, each item.
\begin{itemize}[noitemsep]
	\item \textbf{aligned address requirements (3.1.3.3
		Addresses)}\newline
	System dependent. You can change the default address alignment
	by defining FICL\_ALIGN on your compiler's command line, or in
	platform.h. The default value is set to 2 in ficl.h. This causes
	dictionary entries and ALIGN and ALIGNED to align on 4 byte
	boundaries. To align on 2\textsuperscript{n} byte boundaries,
	set FICL\_ALIGN to \textbf{n}.

	\item \textbf{behavior of 6.1.1320 EMIT for non-graphic
		characters}\newline
	Depends on target system, C runtime library, and your
	implementation of ficlTextOut().

	\item \textbf{character editing of 6.1.0695 ACCEPT and 6.2.1390
		EXPECT}\newline
	None implemented in the versions supplied in primitives.c.
	Because ficlEvaluate() is supplied a text buffer externally,
	it's up to your system to define how that buffer will be obtained.

	\item \textbf{character set (3.1.2 Character types, 6.1.1320
		EMIT, 6.1.1750 KEY)}\newline
	Depends on target system and implementation of ficlTextOut().

	\item \textbf{character-aligned address requirements (3.1.3.3
		Addresses)}\newline
	Ficl characters are one byte each. There are no alignment
	requirements.

	\item \textbf{character-set-extensions matching characteristics
		(3.4.2 Finding definition names)}\newline
	No special processing is performed on characters beyond
	case-folding. Therefore, extended characters will not match
	their unaccented counterparts.

	\item \textbf{conditions under which control characters match a
		space delimiter (3.4.1.1 Delimiters)}\newline
	Ficl uses the Standard C function isspace() to distinguish
	space characters.

	\item \textbf{format of the control-flow stack (3.2.3.2
		Control-flow stack)}\newline
	Uses the data stack.

	\item \textbf{conversion of digits larger than thirty-five
		(3.2.1.2 Digit conversion)}\newline
	The maximum supported value of BASE is 36. Ficl will fail via
	assertion in function ltoa() of utility.c if the base is found
	to be larger than 36 or smaller than 2. There will be no effect
	if NDEBUG is defined, however, other than possibly unexpected
	behavior.

	\item \textbf{display after input terminates in 6.1.0695 ACCEPT
		and 6.2.1390 EXPECT}\newline
	Target system dependent.

	\item \textbf{exception abort sequence (as in 6.1.0680
		ABORT")}\newline
	Calls ABORT to exit.

	\item \textbf{input line terminator (3.2.4.1 User input
		device)}\newline
	Target system dependent (implementation of outer loop that
	calls ficlEvaluate()).

	\item \textbf{maximum size of a counted string, in characters
		(3.1.3.4 Counted strings, 6.1.2450 WORD)}\newline
	Counted strings are limited to 255 characters.

	\item \textbf{maximum size of a parsed string (3.4.1
		Parsing)}\newline
	Limited by available memory and the maximum unsigned value that
	can fit in a cell (2\textsuperscript{32}-1).

	\item \textbf{maximum size of a definition name, in characters
		(3.3.1.2 Definition names)}\newline
	Ficl stores the first 31 characters of a definition name.

	\item \textbf{maximum string length for 6.1.1345 ENVIRONMENT?,
		in characters}\newline
	Same as maximum definition name length.

	\item \textbf{method of selecting 3.2.4.1 User input
		device}\newline
	None supported. This is up to the target system.

	\item \textbf{method of selecting 3.2.4.2 User output
		device}\newline
	None supported. This is up to the target system.

	\item \textbf{methods of dictionary compilation (3.3 The Forth
		dictionary)}\newline
	Okay, we don't know what this means. If you understand what
	they're asking for here, please call the home office.

	\item \textbf{number of bits in one address unit (3.1.3.3
		Addresses)}\newline
	Target system dependent, either 32 or 64 bits.

	\item \textbf{number representation and arithmetic (3.2.1.1
		Internal number representation)}\newline
	System dependent. Ficl represents a CELL internally as a union
	that can hold a ficlInteger32 (a signed 32 bit scalar value),
	a ficlUnsigned32 (32 bits unsigned), and an untyped pointer. No
	specific byte ordering is assumed.

	\item \textbf{ranges for n, +n, u, d, +d, and ud (3.1.3
		Single-cell types, 3.1.4 Cell-pair types)}\newline
	System dependent. Assuming a 32 bit implementation, range for
	signed single-cell values is [-2\textsuperscript{31},
	2\textsuperscript{31}-1]. Range for unsigned single cell values
	is [0, 2\textsuperscript{32}-1]. Range for signed double-cell
	values is [-2\textsuperscript{63}, 2\textsuperscript{63}-1].
	Range for unsigned double cell values is [0,
	2\textsuperscript{64}-1].

	\item \textbf{read-only data-space regions (3.3.3 Data
		space)}\newline
	None.

	\item \textbf{size of buffer at 6.1.2450 WORD (3.3.3.6 Other
		transient regions)}\newline
	Default is 255. Depends on the setting of FICL\_PAD\_SIZE in
	ficl.h.

	\item \textbf{size of one cell in address units (3.1.3
		Single-cell types)}\newline
	System dependent, generally 4.

	\item \textbf{size of one character in address units (3.1.2
		Character types)}\newline
	System dependent, generally 1.

	\item \textbf{size of the keyboard terminal input buffer
		(3.3.3.5 Input buffers)}\newline
	This buffer is supplied by the host program. FICL imposes no
	practical limit.

	\item \textbf{size of the pictured numeric output string buffer
		(3.3.3.6 Other transient regions)}\newline
	Default is 255. Depends on the setting of FICL\_PAD\_SIZE in
	ficl.h.

	\item \textbf{size of the scratch area whose address is returned
		by 6.2.2000 PAD (3.3.3.6 Other transient regions)}\newline
	Default is 255. Depends on the setting of FICL\_PAD\_SIZE in
	ficl.h.

	\item \textbf{system case-sensitivity characteristics (3.4.2
		Finding definition names)}\newline
	The FICL dictionary is not case-sensitive.

	\item \textbf{system prompt (3.4 The Forth text interpreter,
		6.1.2050 QUIT)}\newline
	ok\textgreater

	\item \textbf{type of division rounding (3.2.2.1 Integer
		division, 6.1.0100 */, 6.1.0110 */MOD, 6.1.0230 /,
		6.1.0240 /MOD, 6.1.1890 MOD)}\newline
	Symmetric.

	\item \textbf{values of 6.1.2250 STATE when true}\newline
	1.

	\item \textbf{values returned after arithmetic overflow (3.2.2.2
		Other integer operations)}\newline
	System dependent. FICL makes no special checks for overflow.

	\item \textbf{whether the current definition can be found after
		6.1.1250 DOES\textgreater (6.1.0450 :)}\newline
	No. Definitions are unsmudged after ; only, and only then if no
	control structure matching problems have been detected.
\end{itemize}


\section{Ambiguous Conditions}
\begin{itemize}[noitemsep]
	\item \textbf{a name is neither a valid definition name nor a
		valid number during text interpretation (3.4 The Forth
		text interpreter)}\newline
	FICL calls ABORT then prints the name followed by not found.

	\item \textbf{a definition name exceeded the maximum length
		allowed (3.3.1.2 Definition names)}\newline
	FICL stores the first 31 characters of the definition name, and
	uses all characters of the name in computing its hash code. The
	actual length of the name, up to 255 characters, is stored in
	the definition's length field.

	\item \textbf{addressing a region not listed in 3.3.3 Data
		Space}\newline
	No problem: all addresses in FICL are absolute. You can reach
	any 32 bit address in FICL's address space.

	\item \textbf{argument type incompatible with specified input
		parameter, e.g., passing a flag to a word expecting an
		n (3.1 Data types)}\newline
	FICL makes no check for argument type compatibility. Effects of
	a mismatch vary widely depending on the specific problem and
	operands.

	\item \textbf{attempting to obtain the execution token, (e.g.,
		with 6.1.0070 ', 6.1.1550 FIND, etc.) of a definition
		with undefined interpretation semantics}\newline
	FICL returns a valid token, but the result of executing that
	token while interpreting may be undesirable.

	\item \textbf{dividing by zero (6.1.0100 */, 6.1.0110 */MOD,
		6.1.0230 /, 6.1.0240 /MOD, 6.1.1561 FM/MOD, 6.1.1890
		MOD, 6.1.2214 SM/REM, 6.1.2370 UM/MOD, 8.6.1.1820
		M*/)}\newline
	Results are target procesor dependent. Generally, FICL make no
	checks for divide-by-zero. The target processor will probably
	throw an exception.

	\item \textbf{insufficient data-stack space or return-stack
		space (stack overflow)}\newline
	With FICL\_ROBUST (defined in ficl.h) set to a value of 2 or
	greater, most data, float, and return stack operations are
	checked for underflow and overflow.

	\item \textbf{insufficient space for loop-control
		parameters}\newline
	This is not checked, and bad things will happen.

	\item \textbf{insufficient space in the dictionary}\newline
		FICL generates an error message if the dictionary is
		too full to create a definition header. It checks ALLOT
		as well, but it is possible to make an unchecked
		allocation request that will overflow the dictionary.

	\item \textbf{interpreting a word with undefined interpretation
		semantics}\newline
	FICL protects all ANS Forth words with undefined interpretation
	semantics from being executed while in interpret state. It is
	possible to defeat this protection using ' (tick) and EXECUTE
	though.

	\item \textbf{modifying the contents of the input buffer or a
		string literal (3.3.3.4 Text-literal regions, 3.3.3.5
		Input buffers)}\newline
	Varies depending on the nature of the buffer. The input buffer
	is supplied by FICL's host function, and may reside in read-only
	memory. If so, writing the input buffer can ganerate an
	exception. String literals are stored in the dictionary, and are
	writable.

	\item \textbf{overflow of a pictured numeric output
		string}\newline
	In the unlikely event you are able to construct a pictured
	numeric string of more than FICL\_PAD\_LENGTH characters, the
	system will be corrupted unpredictably. The buffer area that
	holds pictured numeric output is at the end of the virtual
	machine. Whatever is mapped after the offending VM in memory
	will be trashed, along with the heap structures that contain it.

	\item \textbf{parsed string overflow}\newline
		FICL does not copy parsed strings unless asked to.
		Ordinarily, a string parsed from the input buffer during
		normal interpretation is left in-place, so there is no
		possibility of overflow. If you ask to parse a string
		into the dictionary, as in SLITERAL, you need to have
		enough room for the string, otherwise bad things may
		happen. This is usually not a problem.

	\item \textbf{producing a result out of range, e.g.,
		multiplication (using *) results in a value too big to
		be represented by a single-cell integer (6.1.0090 *,
		6.1.0100 */, 6.1.0110 */MOD, 6.1.0570, >NUMBER, 6.1.1561
		FM/MOD, 6.1.2214 SM/REM, 6.1.2370 UM/MOD, 6.2.0970
		CONVERT, 8.6.1.1820 M*/)}\newline
	Value will be truncated.

	\item \textbf{reading from an empty data stack or return stack
		(stack underflow)}\newline
	Most stack underflows are detected and prevented if FICL\_ROBUST
	(defined in sysdep.h) is set to 2 or greater. Otherwise, the
	stack pointer and size are likely to be trashed.

	\item \textbf{unexpected end of input buffer, resulting in an
		attempt to use a zero-length string as a name}\newline
	FICL returns for a new input buffer until a non-empty one is
	supplied.
\end{itemize}

The following specific ambiguous conditions are noted in the glossary entries of the relevant words:
\begin{itemize}[noitemsep]
	\item \textbf{\textgreater IN greater than size of input buffer
		(3.4.1 Parsing)}\newline
	Memory corruption will occur — the exact behavior is
	unpredictable because the input buffer is supplied by the host
	program's outer loop.

	\item \textbf{6.1.2120 RECURSE appears after 6.1.1250
		DOES\textgreater}\newline
	It finds the address of the definition before DOES\textgreater

	\item \textbf{argument input source different than current input
		source for 6.2.2148 RESTORE-INPUT}\newline
	Not implemented.

	\item \textbf{data space containing definitions is de-allocated
		(3.3.3.2 Contiguous regions)}\newline
	This is okay until the cells are overwritten with something
	else. The dictionary maintains a hash table, and the table must
	be updated in order to de-allocate words without corruption.

	\item \textbf{data space read/write with incorrect alignment
		(3.3.3.1 Address alignment)}\newline
	Target processor dependent. Consequences include: none (Intel),
	address error exception (68K).

	\item \textbf{data-space pointer not properly aligned
		(6.1.0150 ,, 6.1.0860 C,)}\newline
	Target processor dependent. Consequences include: none (Intel),
	address error exception (68K).

	\item \textbf{less than u+2 stack items (6.2.2030 PICK,
		6.2.2150 ROLL)}\newline
	If FICL\_ROBUST is two or larger, FICL will detect a stack
	underflow, report it, and execute ABORT to exit execution.
	Otherwise the error will not be detected, and memory corruption
	will occur.

	\item \textbf{loop-control parameters not available
		(6.1.0140 +LOOP, 6.1.1680 I, 6.1.1730 J, 6.1.1760 LEAVE,
		6.1.1800 LOOP, 6.1.2380 UNLOOP)}\newline 
	Loop initiation words are responsible for checking the stack
	and guaranteeing that the control parameters are pushed. Any
	underflows will be detected early if FICL\_ROBUST is set to 2 or
	greater. Note however that FICL only checks for return stack
	underflows at the end of each line of text.

	\item \textbf{most recent definition does not have a name
		(6.1.1710 IMMEDIATE)}\newline
	No problem.

	\item \textbf{name not defined by 6.2.2405 VALUE used by
		6.2.2295 TO}\newline
	FICL's version of TO works correctly with words defined with:
	\begin{itemize}[noitemsep]
		\item VALUE
		\item 2VALUE
		\item FVALUE
		\item F2VALUE
		\item CONSTANT
		\item FCONSTANT
		\item 2CONSTANT
		\item F2CONSTANT
		\item VARIABLE
		\item 2VARIABLE
	\end{itemize}
	as well as with all "local" variables.

	\item \textbf{name not found (6.1.0070 ', 6.1.2033 POSTPONE,
		6.1.2510 ['], 6.2.2530 [COMPILE])}\newline
	FICL prints an error message and executes ABORT

	\item \textbf{parameters are not of the same type (6.1.1240 DO,
		6.2.0620 ?DO, 6.2.2440 WITHIN)}\newline
	Not detected. Results vary depending on the specific problem.

	\item \textbf{6.1.2033 POSTPONE or 6.2.2530 [COMPILE] applied to
		6.2.2295 TO}\newline
	The word is postponed correctly.

	\item \textbf{string longer than a counted string returned by
		6.1.2450 WORD}\newline
	FICL stores the first FICL\_COUNTED\_STRING\_MAX - 1 characters
	in the destination buffer. (The extra character is the trailing
	space required by the standard. Yuck.)

	\item \textbf{u greater than or equal to the number of bits in a
		cell (6.1.1805 LSHIFT, 6.1.2162 RSHIFT)}\newline
	Depends on target processor and C runtime library
	implementations of the \textless\textless and
	\textgreater\textgreater operators on unsigned values. For i386,
	the processor appears to shift modulo the number of bits in a cell.

	\item \textbf{word not defined via 6.1.1000 CREATE (6.1.0550
		\textgreater BODY, 6.1.1250 DOES\textgreater ) and words
		improperly used outside 6.1.0490 \textless \# and
		6.1.0040 \# \textgreater (6.1.0030 \# , 6.1.0050 \# S,
		6.1.1670 HOLD, 6.1.2210 SIGN)}\newline
	Undefined. CREATE reserves a field in words it builds for
	DOES\textgreater  to fill in. If you use DOES\textgreater  on a
	word not made by CREATE it will overwrite the first cell of its
	parameter area. That's probably not what you want. Likewise,
	pictured numeric words assume that there is a string under
	construction in the VM's scratch buffer. If that's not the
	case, results may be unpleasant.
\end{itemize}


\section{Locals Implementation-Defined Options}
\begin{itemize}[noitemsep]
	\item \textbf{maximum number of locals in a definition (13.3.3
		Processing locals, 13.6.2.1795 LOCALS|)}\newline
	Default is 64—unused locals are cheap. Change by redefining
	FICL\_MAX\_LOCALS (defined in ficl.h).
\end{itemize}


\section{Locals Ambiguous conditions}
\begin{itemize}[noitemsep]
	\item \textbf{executing a named local while in interpretation
		state (13.6.1.0086 (LOCAL))}\newline
	Locals can be found in interpretation state while in the
	context of a definition under construction. Under these
	circumstances, locals behave correctly. Locals are not visible
	at all outside the scope of a definition.

	\item \textbf{name not defined by VALUE or LOCAL (13.6.1.2295
		TO)}\newline
	See the CORE ambiguous conditions, above (no change).
\end{itemize}


\section{Programming Tools Implementation-Defined Options}
\begin{itemize}[noitemsep]
	\item \textbf{source and format of display by 15.6.1.2194
		SEE}\newline
	SEE de-compiles definitions from the dictionary. FICL words are
	stored as a combination of things:
	\begin{itemize}[noitemsep]
		\item bytecodes (identified as "instructions"),
		\item addresses of native FICL functions, and
		\item arguments to both of the above.
	\end{itemize}
	Colon definitions are decompiled. Branching instructions
	indicate their destination, but target labels are not
	reconstructed. Literals and string literals are so noted, and
	their contents displayed.
\end{itemize}


\section{Search Order Implementation-Defined Options}
\begin{itemize}[noitemsep]
	\item \textbf{maximum number of word lists in the search order
		(16.3.3 Finding definition names, 16.6.1.2197
		SET-ORDER)}\newline
	Defaults to 16. Can be changed by redefining
	FICL\_MAX\_WORDLISTS (declared in ficl.h).

	\item \textbf{minimum search order (16.6.1.2197 SET-ORDER,
		16.6.2.1965 ONLY)}\newline
	Equivalent to FORTH-WORDLIST 1 SET-ORDER
\end{itemize}


\section{Search Order Ambiguous Conditions}
\begin{itemize}[noitemsep]
	\item \textbf{changing the compilation word list (16.3.3
		Finding definition names)}\newline
	FICL stores a link to the current definition independently of
	the compile wordlist while it is being defined, and links it
	into the compile wordlist only after the definition completes
	successfully. Changing the compile wordlist mid-definition will
	cause the definition to link into the new compile wordlist.

	\item \textbf{search order empty (16.6.2.2037 PREVIOUS)}\newline
	FICL prints an error message if the search order underflows, and
	resets the order to its default state.

	\item \textbf{too many word lists in search order (16.6.2.0715
		ALSO)}\newline
	FICL prints an error message if the search order overflows, and
	resets the order to its default state.
\end{itemize}
